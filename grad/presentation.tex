\documentclass[french, a4paper]{beamer}
\usepackage{url}
\usepackage{hyperref}
\hypersetup{%
    hidelinks,
    pdfinfo = {%
        Author = {Paul Mabileau},
        Title = {Soutenance de Mission en Entreprise},
        Subject = {Stage de fin d'études},
        Keywords = {%
            présentation,
            stage,
            ingénieur,
            intégration,
            sécurité,
            OCD,
            Orange Cyberdéfense,
        },
    },
}

\usepackage[french]{babel}
\usepackage[utf8x]{inputenc}
\usepackage[T1]{fontenc}
\usepackage{times}
\usepackage{beamerthemeWarsaw}
\usepackage{caption}
\usepackage{subcaption}
% \usetheme{Berlin}
\usecolortheme{seagull}
\usefonttheme{serif}
\useoutertheme{shadow}
% \useinnertheme{rectangles}

\captionsetup{justification = centering}
\setbeamertemplate{navigation symbols}{}
\setbeamerfont{page number in head/foot}{size = \small}
\setbeamertemplate{footline}{%
    \hspace{0.2cm}
    \vspace{0.2cm}
    \insertframenumber{} / \inserttotalframenumber{}
}

\captionsetup[figure]{labelformat = empty}
\logo{\includegraphics[height = 15mm]{img/logo/tsp.png}}

\AtBeginSection[] {%
    \begin{frame}
        \tableofcontents[
            currentsection,
            sectionstyle = show/shaded,
            subsectionstyle = show/show/hide,
            subsubsectionstyle = hide/hide/hide/hide,
        ]{}
    \end{frame}
}

\AtBeginSubsection[] {%
    \begin{frame}
        \tableofcontents[
            currentsection,
            sectionstyle = show/hide,
            subsectionstyle = show/shaded/hide,
            subsubsectionstyle = show/show/hide/hide,
        ]{}
    \end{frame}
}

\makeatletter
    \newenvironment{nohead}{%
        \setbeamertemplate{headline}[default]
        \def\beamer@entrycode{\vspace*{-\headheight}}
    }{}
\makeatother


\title{Soutenance de stage: Orange Cyberdéfense}
\subtitle{Intégration sécurité}
\author{Paul Mabileau}
\institute{Télécom SudParis}
\date{7 Octobre 2021}



\begin{document}


\begin{nohead}
    \begin{frame}
        \titlepage{}
    \end{frame}
\end{nohead}

\begin{nohead}
    \begin{frame}
        \begin{center}
            {\Large Plan}
        \end{center}
        \tableofcontents[subsubsectionstyle = hide]
    \end{frame}
\end{nohead}


\section{Introduction}
\subsection{Stage}

\begin{frame}
    \frametitle{Intégration}
    \begin{figure}[h!]
        \centering
        \includegraphics[width = \linewidth]{img/misc/integ.png}
        \caption{Intégration de système}%
        \label{fig:misc/integ}
    \end{figure}
\end{frame}

\subsection{Présentation de l'entreprise}

\begin{frame}
    \frametitle{Orange Cyberdéfense}
    \begin{minipage}{0.5\textwidth}
        \begin{itemize}
            \item Filiale d'Orange Business Services et du groupe Orange.
            \item Fondée en 2014, marque créée en 2016, 1100 employés.
            \item Fournit divers services de sécurité informatique.
        \end{itemize}
    \end{minipage}%
    \hfill
    \begin{minipage}{0.5\textwidth}
        \begin{figure}[h!]
            \centering
            \includegraphics[width = \linewidth]{img/logo/ocd.png}
            \caption{Logo de l'entreprise}%
            \label{fig:logo/ocd}
        \end{figure}
    \end{minipage}
\end{frame}


\section{Mission}
\subsection{Objectifs}

\begin{frame}
    \frametitle{Objectifs}
    \begin{itemize}
        \item Apprendre en auto-formation.
        \item Concevoir et implémenter des architectures sécurisées.
        \item Élaborer des documentations techniques.
        \item Réaliser des projets pour des clients.
        \item Contribuer à des activités internes.
    \end{itemize}
\end{frame}

\subsection{Premières missions}

\begin{frame}
    \frametitle{Auto-formation}
    \begin{minipage}{0.6\textwidth}
        \begin{itemize}
            \item Montée en compétences sur les outils et technologies que
                j'allais utiliser par la suite.
            \item Fortinet:
            \begin{itemize}
                \item FortiGate,
                \item FortiManager,
                \item FortiAnalyzer;
            \end{itemize}
            \item Palo Alto Networks: PAN OS\@;
            \item Microsoft Azure.
        \end{itemize}
    \end{minipage}%
    \hfill
    \begin{minipage}{0.4\textwidth}
        \begin{figure}[h!]
            \centering
            \includegraphics[width = 0.8\linewidth]{img/misc/self-learning.png}%
            \label{fig:misc/self-learning}
        \end{figure}
    \end{minipage}
\end{frame}

\begin{frame}
    \frametitle{Documentation de la salle de tests}
    \begin{minipage}{0.5\textwidth}
        \begin{itemize}
            \item Petite salle serveur pour des essais et de la pré-production.
            \item Revue des baies et équipements internes.
            \item Étiquetage des équipements et des câbles.
            \item Ajout de noms d'hôte et de descriptions d'interfaces.
            \item Inventaire complet des baies et de la réserve.
        \end{itemize}
    \end{minipage}%
    \hfill
    \begin{minipage}{0.45\textwidth}
        \begin{figure}[h!]
            \centering
            \includegraphics[width = \linewidth]{img/misc/data-center.jpg}
            \caption{Image non contractuelle}%
            \label{fig:misc/data-center}
        \end{figure}
    \end{minipage}
\end{frame}

\subsection{Deuxième mission: projet SNCF}

\begin{frame}
    \frametitle{Projet SCNF Access}
    \begin{itemize}
        \item Projet démarré début 2019 et en fin lorsque je le rejoins.
        \item Zone tampon entre l'Internet public et des services SNCF\@.
        \item Analyse avancée du trafic: pare-feu, WAF, IPS\@.
        \item Redondance: \textit{reverse proxy}, GRE, DNS, VIPs, liens croisés.
    \end{itemize}
\end{frame}

\begin{nohead}
    \begin{frame}
        \frametitle{Architecture}
        \begin{figure}[h!]
            \centering
            \includegraphics[height = 1.1\textheight]{img/sncf-access/arch.png}%
            \label{fig:sncf-access/arch}
        \end{figure}
    \end{frame}
\end{nohead}

\begin{frame}
    \frametitle{La mission}
    \begin{itemize}
        \item Rédaction d'un Dossier d'exploitation (DEX).
        \item Plutôt guide d'implémentation de nouveaux services.
        \item Détail de toutes les actions à effectuer à chaque couche.
        \item Manipulation des infrastructures FortiGate et PAN OS en production
            pour aider à la rédaction.
    \end{itemize}
\end{frame}

\subsection{Mission principale: projet Hynamics}
\subsubsection{Considérations générales}

\begin{frame}
    \frametitle{Considérations générales}
    \begin{itemize}
        \item Filiale d'EDF fondée en avril 2019.
        \item Production de dihydrogène renouvelable et à basse émission de
            $CO_2$.
        \item Besoin de sécurité sur leur nouveaux sites: contact d'OCD\@.
        \item Sites concernés:
        \begin{itemize}
            \item La Défense (DEF): <<quartier général>>,
            \item Fontainebleau, <<Les Renardières>> (RNRD): plate-forme de
                test,
            \item Auxerre (AUXR): site de production.
        \end{itemize}
    \end{itemize}
\end{frame}

\begin{frame}
    \frametitle{Architecture globale}
    \begin{figure}[h!]
        \centering
        \includegraphics[width = 0.95\linewidth]{img/doc-hy/global-arch.png}
        \caption{Vue d'ensemble du projet}%
        \label{fig:doc-hy/global-arch}
    \end{figure}
\end{frame}

\begin{frame}
    \frametitle{Architecture physique}
    \begin{figure}[h!]
        \centering
        \includegraphics[width = 0.9\linewidth]{img/doc-hy/site-phys-arch.png}
        \caption{Pour le site de Fontainebleau}%
        \label{fig:doc-hy/site-phys-arch}
    \end{figure}
\end{frame}

\begin{frame}
    \frametitle{Architecture logique}
    \begin{figure}[h!]
        \centering
        \includegraphics[width = 0.9\linewidth]{img/doc-hy/site-logi-arch.png}
        \caption{Pour le site d'Auxerre}%
        \label{fig:doc-hy/site-logi-arch}
    \end{figure}
\end{frame}

\subsubsection{Démarrage et premières opérations}
\subsubsection{Collecte des informations et déploiement sur sites}
\subsubsection{Implémentation de la configuration}
\subsubsection{Mise en place des tests et migration}
\subsubsection{Transferts de connaissances}


\section{Bilan}
\subsection{Retour d'expérience}


\section{Conclusion}
\subsection{Perspectives}


\end{document}
