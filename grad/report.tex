\documentclass[12pt, oneside, a4paper, titlepage]{report}
\usepackage{url}
\usepackage[hidelinks]{hyperref}
\usepackage{nameref}
\usepackage{csquotes}
\usepackage{fancyvrb}
\usepackage[toc,acronym]{glossaries}
\usepackage[export]{adjustbox}
\usepackage{booktabs}
\usepackage{longtable}
\usepackage{tabularx}
\usepackage{pbox}
\usepackage[main=french]{babel}
\usepackage{moresize}
\usepackage{xcolor}
\usepackage{soulutf8}
\usepackage[utf8]{inputenc}
\usepackage[T1]{fontenc}
\usepackage{lmodern}
\usepackage{graphicx}
\usepackage{calc}
\usepackage{geometry}

\geometry{%
    paper=a4paper,  % Paper size, change to letterpaper for US letter size.
    top=3cm,  % Top margin.
    bottom=3cm,  % Bottom margin.
    left=3.5cm,  % Left margin.
    right=3.5cm,  % Right margin.
    headheight=0.75cm,  % Header height.
    footskip=1cm,  % Space from the bottom margin to the baseline of the footer.
    headsep=0.5cm,  % Space from the top margin to the baseline of the header.
    %showframe,  % Uncomment to show how the type block is set on the page.
}

\usepackage[sorting=none]{biblatex}
\addbibresource{refs.bib}
\setacronymstyle{long-short-desc}
\loadglsentries{glossary.tex}
\makeglossaries{}

\newcommand{\nomPrenom}{{%
    \fontsize{14}{14}\selectfont Paul Mabileau
}}
\newcommand{\nomEntreprise}{{%
    \fontsize{16}{16}\selectfont ORANGE CYBERDÉFENSE
}}

\newcommand{\nomAdresseEntreprise}{{%
    \fontsize{12}{12}\selectfont Orange Cyberdéfense, 2 rue Christophe Colomb,
    91300 Massy.
}}

\newcommand{\titreMission}{{%
    \fontsize{14}{14}\selectfont Intégration sécurité: déploiement d'une
    architecture SD-WAN pour trois sites de production.
}}

\newcommand{\nomDirecteurStage}{{%
    \fontsize{12}{12}\selectfont Valentin Goron
}}

\newcommand{\nomConseillerStage}{{%
    \fontsize{12}{12}\selectfont Olivier Paul
}}

\newcommand{\anneeUniversitaire}{{%
    \fontsize{15}{15}\selectfont 2020/2021
}}

\newcommand{\dateDebut}{{%
    \fontsize{12}{12}\selectfont 22/02/2021
}}

\newcommand{\dateFin}{{%
    \fontsize{12}{12}\selectfont 20/08/2021
}}

\newcommand{\vap}{{%
    \fontsize{11}{11}\selectfont SSR
}}



\begin{document}

\begin{titlepage}
    \newpage
\thispagestyle{empty}


\begingroup
    \begin{tabular}{ll}
        % LOGO DE TELECOM SUDPARIS
        \begin{minipage}[c]{0.0\textwidth}
            \begin{flushleft}
                \includegraphics[width=3cm]{img/tsp.png}
            \end{flushleft}
        \end{minipage}
        &
        % ANNEE UNIVERSITAIRE
        \hspace{-3em}
        \begin{minipage}[c]{1.0\textwidth}
            \begin{flushright}
                \textbf{ANNÉE \anneeUniversitaire}
            \end{flushright}
        \end{minipage}
        \hspace{\stretch{1}}
    \end{tabular}
    \vskip 4.0em
\endgroup

\begingroup
    \begin{center}
        \Large{\textbf{RAPPORT de <<MISSION en ENTREPRISE>>}}
        \vskip 2.0em

        présenté par:
        \vskip 2.0em
        \textbf{\nomPrenom}

        \vskip 3.0em
        Mission effectuée de \dateDebut{} au \dateFin{} chez
        \vskip 2.0em
        \textbf{\nomEntreprise}
        \vskip 3.5em

        Sujet de la mission:
    \end{center}

     % CADRE POUR LE TITRE DE LA MISSION
    \parbox[b]{1.0\textwidth-1cm}{%
        \hrule height 1.5pt
        \vrule width 1.5pt
        \hspace{1.80cm}
        \parbox[b]{\textwidth-5cm}{%
            \vskip 0.6em
            \center\large{\textbf{\titreMission}}\endcenter{}
            \vskip 0.6em
        }
        \hspace{1.80cm}
        \vrule width 1.5pt
        \hrule height 1.5pt
    }
    \vskip 3.5em
\endgroup

\begingroup
    Directeur de stage: \textbf{\nomDirecteurStage}
    \vskip 1.0em
    Conseiller de Stage: \textbf{\nomConseillerStage{}}
\endgroup

\vskip 4.5em

\begingroup
    \hrule height 1.0pt
    \vskip 0.5em
    \noindent Travail effectué pour la société: \nomAdresseEntreprise{}
\endgroup


\newpage

\end{titlepage}

\tableofcontents

\newpage
\thispagestyle{empty}


\begingroup
    \begin{tabular}{ll}
    % LOGO DE TELECOM SUDPARIS
        \begin{minipage}[c]{0.0\textwidth}
            \begin{flushleft}
                \includegraphics[width=4cm]{img/tsp.png}
            \end{flushleft}
        \end{minipage}
        &
        % ANNEE UNIVERSITAIRE
        \hspace{-3em}
        \begin{minipage}[c]{1.0\textwidth}
            \begin{flushright}
                \Large{\textbf{Stage de 3\up{ème} année}}\\
                \Large{\textbf{\anneeUniversitaire}}
            \end{flushright}
        \end{minipage}
        \hspace{\stretch{1}}
    \end{tabular}
    \vskip 1.0em
\endgroup

\begingroup
    \noindent \textbf{Nom de l'étudiant}: \nomPrenom{}
    \vskip 0.5em
    \noindent \textbf{VAP}: \vap{}
    \vskip 0.5em
    \noindent \textbf{Entreprise}: \nomEntreprise{}
    \vskip 0.5em
    \noindent \textbf{Dates}: du \dateDebut{} au \dateFin{}
    \vskip 0.5em
    \noindent \textbf{Sujet}: \titreMission{}
    \vskip 3em

    % FR
    \noindent
    \fbox{%
        \parbox{\textwidth}{%
            \vskip 0.8em
            \hspace{0.5cm}
            \vspace{0.4em}
            \large{\textbf{RÉSUMÉ:}}
            \vskip 0.4em
            \vskip 0.8em
        }
    }

    % EN
    \noindent
    \fbox{%
        \centering
        \parbox{\textwidth}{%
            \vskip 0.8em
            \vspace{0.4em}
            \hspace{0.5cm}
            \large{\textbf{ABSTRACT:}}
            \vskip 0.4em
            \vskip 0.8em
        }
    }
    \endgroup


\newpage


\chapter{Remerciements}%
\label{cha:ackn}

Je tiens tout d'abord à remercier mon directeur de stage, Valentin Goron, pour
son suivi régulier du stage et des activités que j'y menais, pour souvent
m'orienter vers une personne prête à m'aider à résoudre quelconque problème. Je
remercie donc Guerric Lemoigne pour son accueil efficace et toutes les petites
opérations et questions qui viennent avec un début de stage, Andrès Marino pour
les dépannages en salle serveur, Arnaud Fauvel pour la rédaction du DEX, Deroi
Tieffi pour avoir été un bon taxi, Geoffrey Giganti pour le coup de pouce pour
démarrer en Stormshield, et j'en oublie d'autres.

Je remercie les chefs de projet avec qui j'ai eu l'occasion de collaborer, ne
serait-ce qu'un peu: Arnaud Creuser et surtout Laurent Saunier pour avoir
brillament géré un projet en mouvement constant, bourrés de changements de
dernière minute et fréquemment bloqué ou retardé par des imprévus sortant de nos
responsabilités.

Merci aussi à Gauthier pour les longues journées de nos débuts de stage passées
en salle de test à documenter, nettoyer, réorganiser et étiqueter tout ce qui
bouge --- ou en l'occurence plutôt pas trop.

Enfin, je tiens tout particulièrement à remercier Sylvain Bardy pour un travail
en binôme qu'il rend toujours efficace mais surtout agréable du fait de son
soutien constant et de sa bonne humeur inconditionnelle. Même lorsque contraint
à devoir jongler entre plusieurs projets dont d'autres bien plus prenant, il
était toujours prêt à répondre à mes nombreuses questions et interrogations,
voire à se libérer d'une minute à l'autre pour résoudre divers problèmes
ensemble sans pour autant le demander. Je pense sincèrement que la réussite du
projet en question et de la fin de mon stage peut lui être attribuée en grande
partie. Merci beaucoup.

\chapter{Introduction}%
\label{cha:intro}

\section{Stage}%
\label{sec:intro-stage}

Aujourd'hui, la sécurité est souvent citée comme un enjeu majeur pour les
entreprises ainsi que pour l'ensemble des acteurs qui l'entourent. Elle n'est
plus confinée uniquement au rôle de l'informaticien. Le domaine devient de plus
en plus important à cause de la transition progressive vers des systèmes
informatiques~\cite{reliance}, de l'Internet et de protocoles réseau sans fil
tel que Bluetooth ou Wi-Fi, et de la constante augmentation du nombre
d'appareils <<intelligents>> tels que téléphones et télévisions modernes et les
nombreux types différents d'appareils constituant <<l'Internet des objets>>. Du
fait de sa complexité grandissante, à la fois en termes politiques et
technologiques, la cybersécurité est aussi un des défis proéminents du monde
contemporain~\cite{global-cyber}. Ces raisons amènent alors de nombreuses
entreprises à investir un temps et un budget significatif à l'établissement
d'une sécurité opérationnelle éstimée adaptée aux contraintes liées à leur cœur
de métier. C'est dans ce cadre que s'incrit l'activité d'Orange Cyberdéfense et
en particulier mon stage.

Mon stage avait pour domaine principal l'\gls{integ} d'outils de sécurité pour
le compte de clients d'Orange Cyberdéfense. Comme ceux-ci sont variés, les
projets associés abordent des sujets plus ou moins spécifiques, des enjeux et
contraintes différents, une criticité de la sécurité adaptée, des durées et
budgets variés, \ldots L'idée est donc de les accompagner dans une démarche
d'identification de sujets et problèmes de sécurité les concernant directement,
d'amélioration de leurs infrastructures, logiciels et configurations, et de
conception d'architectures adaptées à l'implémentation de ces changements dans
le cadre d'un projet dédié.

L'\gls{integ} de système est une activité d'ingénierie qui consiste en le
processus de mise en relation de différents systèmes, souvent disparates, dans
le but qu'ils deviennent composants d'un nouveau système plus global. En
informatique, il peut s'agir de connexion physiques entre machines ou logiques
entre logiciels. En sécurité, cela ajoute certaines contraintes, notamment de ne
pas s'arrêter uniquement au bon fonctionnement du système global mais d'inclure
aussi le non-fonctionnement souhaité de ce qui n'appartient pas au système.


\section{Présentation de l'entreprise}%
\label{sec:intro-entreprise}

\begin{figure}[h!]
    \centering
    \includegraphics[width=0.8\linewidth]{img/logo/ocd.png}
    \caption{Logo de la marque \acrlong{ocd}}%
    \label{fig:logo/ocd}
\end{figure}

\Gls{ocd} SAS est une filiale d'\gls{obs}, elle-même filiale et marque
commerciale du groupe Orange. \gls{ocd} est spécialisée dans la prestation de
services en cybersécurité: elle accompagne les entreprises dans la sécurisation
de leurs activités et de leurs données. L'entreprise a son siège dans le
quartier de La Défense, en région parisienne et compte plus de 2500
collaborateurs dont plus de 250 chercheurs et analystes à travers le monde,
répartis dans 160 pays~\cite{ocd}.

\acrlong{ocd} a été fondée en 2014 et dirigé jusqu'en avril 2021 par Michel Van
Den Berghe. La marque est créée en 2016, regroupant les activités en
cybersécurité d'Orange Consulting et d'Atheos, cabinet de conseil racheté en
2014~\cite{rachat-atheos}.  La filiale a ensuite évoluée avec le rachat de Lexsi
en 2016 puis SecureData et SecureLink en 2019 pour une valeur de 515 millions
d'euros~\cite{rachat-securelink}. La direction est reprise en avril 2021 par
Hugues Foulon.

Financièrement et humainement, l'entreprise a connu une forte croissance depuis
sa création. Entre 2016 et 2020, \gls{ocd} a eu une croissance de chiffre
d'affaires de 21,3\% en moyenne, pour atteindre 300 millions d'euros en 2020 et
768 millions d'euros en 2020 aussi en prenant en compte toutes les acquisitions
de la filiale~\cite{ocd}. Le résultat net quant à lui a quelque peu oscillé au
fur et à mesure des années, mais était par exemple de 2,984 millions d'euros en
2016, 883 milliers d'euros en 2018 et 1,826 millions d'euros en 2020. L'effectif
salarial est lui aussi en augmentation assez régulière depuis 2015 avec en
moyenne 30,0\% de croissance jusqu'à fin 2020, pour atteindre plus de 1100
employés début 2021~\cite{finances-ocd}. L'ensemble du détail des finances de
l'entreprise est mis à disposition en annexe~\ref{tab:finances-ocd}.

\acrlong{ocd} propose une notable variété de services informatiques en sécurité.
Ils contiennent en particulier au niveau purement technique: du conseil, de
l'audit organisationnel et de sécurité par tests d'intrusion, de la détection de
menaces de manière transparente aux clients et des réponses automatisées, de la
conception d'architectures nouvelles ou en vue d'une migration avec de
l'\gls{integ}, et enfin du support et du \gls{soc} pour atteindre un \gls{mcs}.
Il y a aussi des services plus orienté sur l'axe humain de la sécurité: des
formations visant à sensibiliser les employées d'une entreprise cliente, mais
aussi de l'assistance à la gestion de crise~\cite{ocd}.  Puisque je n'ai fait
partie que de l'équipe d'\gls{integ}, je ne serais en mesure de préciser la
structure interne et le fonctionnement précis des autres équipes. Leur
intervention dans le déroulé d'un projet complet sera cependant abordé en de
plus amples détails dans la partie~\ref{cha:mission} --- \nameref{cha:mission}.

En particulier, l'offre de filtrage délocalisé et réponse automatisée à incident
est réalisée grâce à des infrastructures que l'entreprise a conçues et maintient
soi-même. \acrlong{ocd} héberge au sein de 26 centres de détection dans 13 pays
du monde de quoi analyser en moyenne 50 milliards d'évènements par jour, ce qui
permet notamment de révéler et rapidement clotûrer en moyenne plus de 200 sites
Web détectés malveillants par jour.

\section{Acteurs du marché}%
\label{sec:intro-acteurs}

La sécurité informatique est un enjeu majeur pour toute entreprise et surtout
constant dans le temps. En effet, la migration vers une informatique généralisée
rend la surface d'attaque en général de plus en plus importante. Les attaques
informatiques sont réalisées en permanence et sont surtout de types très variés.

Rien que dans les ménages français par exemple, les personnes faisant face à des
problèmes de sécurité rencontrent huit catégories d'attaques différentes et
déclarent dans 42,5\% des cas avoir été sujet à du hameçonnage par réception
d'un message invitant à se connecter à un site Web frauduleux et dans 21,4\% des
cas par redirection vers un site frauduleux invitant à fournir des informations
personnelles lors d'une navigation Web~\cite{attack-types}.

Cela concerne tout aussi bien les entreprises du secteur privés: 16\% des
sociétés de 10 personnes ou plus implantées en France déclarent avoir vécu un
incident de sécurité informatique en 2018. Les sociétés de 250 personnes ou plus
sont deux fois plus touchées~\cite{companies-security}. Le simple coût lié à la
gestion de la crise déclenchée par une attaque et aux réparations qui
s'ensuivent est conséquent: il est aux États-Unis en moyenne de 27,4 millions de
dollars et en France de 9,72 millions de dollars USD~\cite{attack-costs}. Cela
les amène alors à investir un budget et un temps considérable pour se protéger
en conséquence: en 2019, 87\% de sociétés de 10 personnes ou plus réalisent des
activités en lien avec la sécurité de leur système d'information. Par ailleurs,
67\% des sociétés ont recours à des prestataires pour réaliser des activités de
sécurité informatique, parfois en plus des employés de l'entreprise, et 20\% les
font réaliser uniquement par leurs propres employés, tandis que les grandes
entreprises uniquement le font à 31\%. Ces activités comprennent notamment de
l'édition de communications internes: 26\% des sociétés ont une documentation
sur les mesures, pratiques ou procédures en matière de sécurité des systèmes
d'information, autant qu'en 2015. C'est le cas de 71\% des sociétés de 250
personnes ou plus. Quelle que soit leur taille, sept sociétés sur dix ont défini
ou révisé cette documentation au cours de l'année
écoulée~\cite{companies-security}.

Cette masse d'entreprises ayant recours à des prestataires de services externes
pour les aider à réaliser toutes ces activités particulièrement chronophages est
alors très importante. Le marché associé est donc évidemment conséquent: en
2020, il a représenté à l'échelle globale environ 133 milliards de dollars
USD~\cite{security-market}, dont 96,3 en services de
consultation~\cite{security-consulting-market}, ce qui inclus la grande partie
des activités d'\gls{ocd} par exemple. \acrlong{ocd} est en plus de cela sur la
plupart des fronts de la sécurité (voir la section
précédente~\ref{sec:intro-entreprise}), puisque sert notamment à plus de 8000
clients de tous secteurs d'activité~\cite{ocd}. Son marché direct est alors de
très grande taille et les acteurs y participant en grand nombre. Quant à
certains en particulier, nous citerons par exemple Atos et Wavestone pour la
partie conseil et conception d'architectures, Synacktiv et Quarkslab pour la
partie audit de sécurité et tests d'intrusion, ou encore Login Sécurité en
partie pour les opérations de \gls{soc}.

\chapter{Mission}%
\label{cha:mission}

Ma mission était, d'un point de vue assez global, de participer aux divers
projets et activités de l'équipe d'ingénieurs en \gls{integ} faisant partie du
département <<Solutions de confiance>> d'\acrlong{ocd}. Cette équipe est
elle-même composée de deux sous-équipes: le \gls{ocd-build} et le \gls{ocd-run}.
Le \gls{ocd-build} est chargé d'une partie de la conception de l'architecture à
mettre en place pour répondre aux besoins d'un client, puis surtout de son
implémentation effective qui nécessite souvent une intervention matérielle et
l'établissement de configurations logicielles cibles, pour enfin réaliser une
migration vers le nouveau système. Le \gls{ocd-run} quant à lui est chargé
d'assurer le bon fonctionnement du système implémenté par le \gls{ocd-build} de
sorte qu'il soit opérationnel la majeure partie du temps, de surveiller la
sécurité du parc informatique en question et donc réaliser le rôle de \gls{soc},
mais aussi de faire évoluer les configurations logicielles au fur et à mesure
des demandes du client par voie de tickets. J'ai, pour ma part, rejoint en
particulier l'équipe \gls{ocd-build} pour la durée entière de ce stage.

\section{Objectifs}%
\label{sec:mission-objectifs}

\section{Relations humaines et management}%
\label{sec:mission-rhm}

\chapter{Responsabilité sociétale des entreprises}%
\label{cha:rse}

\section{Economie}%
\label{sec:rse-eco}

\section{Social}%
\label{sec:rse-social}

\section{Environnement}%
\label{sec:rse-env}

\section{Gouvernance}%
\label{sec:rse-gouv}

\section{Santé et Sécurité au Travail}%
\label{sec:rse-sst}

\chapter{Bilan}%
\label{cha:bilan}

\section{Retour d'expérience}%
\label{sec:bilan-retexp}

\chapter{Conclusion}%
\label{cha:conclu}

\section{Perspectives}%
\label{sec:conclu-persp}

\chapter{Annexes}%
\label{cha:annexes}

\begin{longtable}{|l|r|r|r|r|}
    \hline
    \textbf{Performance} & \textbf{2020} & \textbf{2019} & \textbf{2018} &
    \textbf{2017} \\ \hline
        Chiffre d'affaires (\texteuro) & 301M & 274M & 219M & 191M \\ \hline
        Marge brute (\texteuro) & 204M & 179M & 153M & 126M \\ \hline
        EBITDA --- EBE (\texteuro) & 12,5M & 13,3M & 8,97M & 5,94M \\ \hline
        Résultat d'exploitation (\texteuro)
            & 3,63M & 4,8M & 1,43M & 1,2M \\ \hline
        Résultat net (\texteuro) & 1,83M & 2,13M & 883K & 1,39M \\ \hline
    \textbf{Croissance} & \textbf{2020} & \textbf{2019} & \textbf{2018} &
    \textbf{2017} \\ \hline
        Taux de croissance du CA (\%) & 9,8 & 25,1 & 14,8 & 35,6 \\ \hline
        Taux de marge brute (\%) & 67,7 & 65,2 & 69,7 & 65,8 \\ \hline
        Taux de marge d'EBITDA (\%) & 4,2 & 4,8 & 4,1 & 3,1 \\ \hline
        Taux de marge opérationnelle (\%) & 1,2 & 1,8 & 0,7 & 0,6 \\ \hline
    \textbf{Gestion BFR} & \textbf{2020} & \textbf{2019} & \textbf{2018} &
    \textbf{2017} \\ \hline
        BFR (\texteuro) & 38,4M & 40,8M & 2,07M & 7,33M \\ \hline
        BFR exploitation (\texteuro) & 55,8M & 69,9M & 55,6M & 58,2M \\ \hline
        BFR hors exploitation (\texteuro) & -17,4M
            & -29,1M & -53,5M & -50,9M \\ \hline
        BFR (j de CA) & 46,6 & 54,3 & 3,4 & 14 \\ \hline
        BFR exploitation (j de CA) & 67,7 & 93,1 & 92,6 & 111 \\ \hline
        BFR hors exploitation (j de CA) & -21,1 & -38,8 & -89,2 & -97,4 \\ \hline
        Délai de paiement clients (j) & 168 & 194 & 176 & 199 \\ \hline
        Délai de paiement fournisseurs (j) & 173 & 175 & 140 & 142 \\ \hline
        Ratio des stocks / CA (j) & 12,9 & 16,7 & 10,9 & 7,7 \\ \hline
    \textbf{Autonomie financière} & \textbf{2020} & \textbf{2019} &
    \textbf{2018} & \textbf{2017} \\ \hline
        Capacité d'autofinancement (\texteuro)
            & 10,8M & 10,6M & 8,27M & 5,28M \\ \hline
        Capacité d'autofinancement / CA (\%) & 3,6 & 3,9 & 3,8 & 2,8 \\ \hline
        Fonds de roulement net global (\texteuro)
            & 41,2M & 43M & 6,02M & 10,9M \\ \hline
        Couverture du BFR & 1,1 & 1,1 & 2,9 & 1,5 \\ \hline
        Trésorerie (\texteuro) & 2,96M & 2,31M & 3,98M & 3,48M \\ \hline
        Dettes financières (\texteuro) & 40M & 50,8M & 16M & 21M \\ \hline
        Capacité de remboursement & 3,4 & 4,6 & 1,5 & 3,3 \\ \hline
        Ratio d'endettement (Gearing) & 0,5 & 0,7 & 0,2 & 0,3 \\ \hline
        Autonomie financière (\%) & 22,9 & 22,8 & 29,5 & 30,2 \\ \hline
        Taux de levier (DFN/EBITDA) & 3 & 3,7 & 1,3 & 3 \\ \hline
    \textbf{Solvabilité} & \textbf{2020} & \textbf{2019} & \textbf{2018} &
    \textbf{2017} \\ \hline
        Etat des dettes à 1 an au plus (\texteuro)
            & 199M & 182M & 142M & 131K \\ \hline
        Liquidité générale & 1,2 & 1,2 & 1 & 1,04K \\ \hline
        Couverture des dettes & 2,3 & 1,9 & 7,1 & 4,6 \\ \hline
    \textbf{Rentabilité} & \textbf{2020} & \textbf{2019} & \textbf{2018} &
    \textbf{2017} \\ \hline
        Marge nette (\%) & 0,6 & 0,8 & 0,4 & 0,7 \\ \hline
        Rentabilité sur fonds propres (\%) & 2,6 & 3,2 & 1,3 & 2,1 \\ \hline
        Rentabilité économique (\%) & 0,6 & 0,7 & 0,4 & 0,6 \\ \hline
        Valeur ajoutée (\texteuro) & 103M & 94,2M & 77M & 62,6M \\ \hline
        Valeur ajoutée / CA (\%) & 34,1 & 34,4 & 35,1 & 32,8 \\ \hline
    \textbf{Structure d'activité} & \textbf{2020} & \textbf{2019} &
    \textbf{2018} & \textbf{2017} \\ \hline
        Salaires et charges sociales (\texteuro)
            & 89,2M & 78,9M & 66,6M & 55,3M \\ \hline
        Salaires / CA (\%) & 29,6 & 28,8 & 30,4 & 29 \\ \hline
        Impôts et taxes (\texteuro) & 4,11M & 4,09M & 3,27M & 2,66M \\ \hline
    \caption{Résultats financiers~\cite{finances-ocd}}%
    \label{tab:finances-ocd}
\end{longtable}

\printbibliography[title=Références]

\listoffigures

\glsaddall{}
\printglossary[type=main]
\printglossary[type=\acronymtype, title=Acronymes]

\end{document}
