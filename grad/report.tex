\documentclass[12pt, oneside, a4paper, titlepage]{report}
\usepackage{url}
\usepackage{hyperref}
\hypersetup{%
    hidelinks,
    pdfinfo = {%
        Author = {Paul Mabileau},
        Title = {Rapport de Mission en Entreprise},
        Subject = {Stage de fin d'études},
        Keywords = {%
            rapport,
            stage,
            ingénieur,
            intégration,
            sécurité,
            OCD,
            Orange Cyberdéfense,
        },
    },
}

\usepackage{nameref}
\usepackage{fancyvrb}
\usepackage[toc, acronym]{glossaries}
\usepackage[export]{adjustbox}
\usepackage{booktabs}
\usepackage{longtable}
\usepackage{tabularx}
\usepackage{pbox}
\usepackage[main = french]{babel}
\usepackage{moresize}
\usepackage{xcolor}
\usepackage{soulutf8}
\usepackage[utf8]{inputenc}
\usepackage[T1]{fontenc}
\usepackage{csquotes}
\usepackage{lmodern}
\usepackage{graphicx}
\usepackage[section]{placeins}
\usepackage{calc}

\usepackage{geometry}
\geometry{%
    paper = a4paper,  % Paper size, change to letterpaper for US letter size.
    top = 3cm,  % Top margin.
    bottom = 3cm,  % Bottom margin.
    left = 3.5cm,  % Left margin.
    right = 3.5cm,  % Right margin.
    headheight = 0.75cm,  % Header height.
    footskip = 1cm,  % Space from the bottom margin to the baseline of the footer.
    headsep = 0.5cm,  % Space from the top margin to the baseline of the header.
    %showframe,  % Uncomment to show how the type block is set on the page.
}

\usepackage[sorting = none]{biblatex}
\addbibresource{refs.bib}
\setacronymstyle{long-short-desc}
\loadglsentries{glossary.tex}
\makeglossaries{}

\newcommand{\nomPrenom}{{%
    \fontsize{14}{14}\selectfont Paul Mabileau
}}
\newcommand{\nomEntreprise}{{%
    \fontsize{16}{16}\selectfont ORANGE CYBERDÉFENSE
}}

\newcommand{\nomAdresseEntreprise}{{%
    \fontsize{12}{12}\selectfont Orange Cyberdéfense, 2 rue Christophe Colomb,
    91300 Massy.
}}

\newcommand{\titreMission}{{%
    \fontsize{14}{14}\selectfont Intégration sécurité: déploiement d'une
    architecture SD-WAN pour trois sites de production.
}}

\newcommand{\nomDirecteurStage}{{%
    \fontsize{12}{12}\selectfont Valentin Goron
}}

\newcommand{\nomConseillerStage}{{%
    \fontsize{12}{12}\selectfont Olivier Paul
}}

\newcommand{\anneeUniversitaire}{{%
    \fontsize{15}{15}\selectfont 2020/2021
}}

\newcommand{\dateDebut}{{%
    \fontsize{12}{12}\selectfont 22/02/2021
}}

\newcommand{\dateFin}{{%
    \fontsize{12}{12}\selectfont 20/08/2021
}}

\newcommand{\vap}{{%
    \fontsize{11}{11}\selectfont SSR
}}




\begin{document}

\begin{titlepage}
    \newpage
\thispagestyle{empty}


\begingroup
    \begin{tabular}{ll}
        % LOGO DE TELECOM SUDPARIS
        \begin{minipage}[c]{0.0\textwidth}
            \begin{flushleft}
                \includegraphics[width=3cm]{img/tsp.png}
            \end{flushleft}
        \end{minipage}
        &
        % ANNEE UNIVERSITAIRE
        \hspace{-3em}
        \begin{minipage}[c]{1.0\textwidth}
            \begin{flushright}
                \textbf{ANNÉE \anneeUniversitaire}
            \end{flushright}
        \end{minipage}
        \hspace{\stretch{1}}
    \end{tabular}
    \vskip 4.0em
\endgroup

\begingroup
    \begin{center}
        \Large{\textbf{RAPPORT de <<MISSION en ENTREPRISE>>}}
        \vskip 2.0em

        présenté par:
        \vskip 2.0em
        \textbf{\nomPrenom}

        \vskip 3.0em
        Mission effectuée de \dateDebut{} au \dateFin{} chez
        \vskip 2.0em
        \textbf{\nomEntreprise}
        \vskip 3.5em

        Sujet de la mission:
    \end{center}

     % CADRE POUR LE TITRE DE LA MISSION
    \parbox[b]{1.0\textwidth-1cm}{%
        \hrule height 1.5pt
        \vrule width 1.5pt
        \hspace{1.80cm}
        \parbox[b]{\textwidth-5cm}{%
            \vskip 0.6em
            \center\large{\textbf{\titreMission}}\endcenter{}
            \vskip 0.6em
        }
        \hspace{1.80cm}
        \vrule width 1.5pt
        \hrule height 1.5pt
    }
    \vskip 3.5em
\endgroup

\begingroup
    Directeur de stage: \textbf{\nomDirecteurStage}
    \vskip 1.0em
    Conseiller de Stage: \textbf{\nomConseillerStage{}}
\endgroup

\vskip 4.5em

\begingroup
    \hrule height 1.0pt
    \vskip 0.5em
    \noindent Travail effectué pour la société: \nomAdresseEntreprise{}
\endgroup


\newpage

\end{titlepage}

\tableofcontents

\newpage
\thispagestyle{empty}


\begingroup
    \begin{tabular}{ll}
    % LOGO DE TELECOM SUDPARIS
        \begin{minipage}[c]{0.0\textwidth}
            \begin{flushleft}
                \includegraphics[width=4cm]{img/tsp.png}
            \end{flushleft}
        \end{minipage}
        &
        % ANNEE UNIVERSITAIRE
        \hspace{-3em}
        \begin{minipage}[c]{1.0\textwidth}
            \begin{flushright}
                \Large{\textbf{Stage de 3\up{ème} année}}\\
                \Large{\textbf{\anneeUniversitaire}}
            \end{flushright}
        \end{minipage}
        \hspace{\stretch{1}}
    \end{tabular}
    \vskip 1.0em
\endgroup

\begingroup
    \noindent \textbf{Nom de l'étudiant}: \nomPrenom{}
    \vskip 0.5em
    \noindent \textbf{VAP}: \vap{}
    \vskip 0.5em
    \noindent \textbf{Entreprise}: \nomEntreprise{}
    \vskip 0.5em
    \noindent \textbf{Dates}: du \dateDebut{} au \dateFin{}
    \vskip 0.5em
    \noindent \textbf{Sujet}: \titreMission{}
    \vskip 3em

    % FR
    \noindent
    \fbox{%
        \parbox{\textwidth}{%
            \vskip 0.8em
            \hspace{0.5cm}
            \vspace{0.4em}
            \large{\textbf{RÉSUMÉ:}}
            \vskip 0.4em
            \vskip 0.8em
        }
    }

    % EN
    \noindent
    \fbox{%
        \centering
        \parbox{\textwidth}{%
            \vskip 0.8em
            \vspace{0.4em}
            \hspace{0.5cm}
            \large{\textbf{ABSTRACT:}}
            \vskip 0.4em
            \vskip 0.8em
        }
    }
    \endgroup


\newpage



\chapter{Remerciements}%
\label{cha:ackn}

Je tiens tout d'abord à remercier mon directeur de stage, Valentin Goron, pour
son suivi régulier du stage et des activités que j'y menais, pour souvent
m'orienter vers une personne prête à m'aider à résoudre quelconque problème. Je
remercie donc Guerric Lemoigne pour son accueil efficace et toutes les petites
opérations et questions qui viennent avec un début de stage, Andrès Marino pour
les dépannages en salle serveur, Arnaud Fauvel pour la rédaction du DEX, Deroi
Tieffi pour avoir été un bon taxi, Geoffrey Giganti pour le coup de pouce pour
démarrer en Stormshield, et j'en oublie d'autres.

Je remercie les chefs de projet avec qui j'ai eu l'occasion de collaborer, ne
serait-ce qu'un peu: Arnaud Creuser et surtout Laurent Saunier pour avoir
brillament géré un projet en mouvement constant, bourrés de changements de
dernière minute et fréquemment bloqué ou retardé par des imprévus sortant de nos
responsabilités.

Merci aussi à Gauthier pour les longues journées de nos débuts de stage passées
en salle de test à documenter, nettoyer, réorganiser et étiqueter tout ce qui
bouge --- ou en l'occurence plutôt pas trop.

Enfin, je tiens tout particulièrement à remercier Sylvain Bardy pour un travail
en binôme qu'il rend toujours efficace mais surtout agréable du fait de son
soutien constant et de sa bonne humeur inconditionnelle. Même lorsque contraint
à devoir jongler entre plusieurs projets dont d'autres bien plus prenant, il
était toujours prêt à répondre à mes nombreuses questions et interrogations,
voire à se libérer d'une minute à l'autre pour résoudre divers problèmes
ensemble sans pour autant le demander. Je pense sincèrement que la réussite du
projet en question et de la fin de mon stage peut lui être attribuée en grande
partie. Merci beaucoup.


\chapter{Introduction}%
\label{cha:intro}

\section{Stage}%
\label{sec:intro::stage}

Aujourd'hui, la sécurité est souvent citée comme un enjeu majeur pour les
entreprises ainsi que pour l'ensemble des acteurs qui l'entourent. Elle n'est
plus confinée uniquement au rôle de l'informaticien. Le domaine devient de plus
en plus important à cause de la transition progressive vers des systèmes
informatiques~\cite{reliance}, de l'Internet et de protocoles réseau sans fil
tel que Bluetooth ou Wi-Fi, et de la constante augmentation du nombre
d'appareils <<intelligents>> tels que téléphones et télévisions modernes et les
nombreux types différents d'appareils constituant <<l'Internet des objets>>. Du
fait de sa complexité grandissante, à la fois en termes politiques et
technologiques, la cybersécurité est aussi un des défis proéminents du monde
contemporain~\cite{global-cyber}. Ces raisons amènent alors de nombreuses
entreprises à investir un temps et un budget significatif à l'établissement
d'une sécurité opérationnelle éstimée adaptée aux contraintes liées à leur cœur
de métier. C'est dans ce cadre que s'incrit l'activité d'Orange Cyberdéfense et
en particulier mon stage.

Mon stage avait pour domaine principal l'\gls{integ} d'outils de sécurité pour
le compte de clients d'Orange Cyberdéfense. Comme ceux-ci sont variés, les
projets associés abordent des sujets plus ou moins spécifiques, des enjeux et
contraintes différents, une criticité de la sécurité adaptée, des durées et
budgets variés, \ldots{} L'idée est donc de les accompagner dans une démarche
d'identification de sujets et problèmes de sécurité les concernant directement,
d'amélioration de leurs infrastructures, logiciels et configurations, et de
conception d'architectures adaptées à l'implémentation de ces changements dans
le cadre d'un projet dédié.

L'\gls{integ} de système est une activité d'ingénierie qui consiste en le
processus de mise en relation de différents systèmes, souvent disparates, dans
le but qu'ils deviennent composants d'un nouveau système plus global. En
informatique, il peut s'agir de connexion physiques entre machines ou logiques
entre logiciels. En sécurité, cela ajoute certaines contraintes, notamment de ne
pas s'arrêter uniquement au bon fonctionnement du système global mais d'inclure
aussi le non-fonctionnement souhaité de ce qui n'appartient pas au système.


\section{Présentation de l'entreprise}%
\label{sec:intro::entreprise}

\begin{figure}[h!]
    \centering
    \includegraphics[width = 0.8\linewidth]{img/logo/ocd.png}
    \caption{Logo de la marque \acrlong{ocd}}%
    \label{fig:logo/ocd}
\end{figure}

\Gls{ocd} SAS est une filiale d'\gls{obs}, elle-même filiale et marque
commerciale du groupe Orange. \gls{ocd} est spécialisée dans la prestation de
services en cybersécurité: elle accompagne les entreprises dans la sécurisation
de leurs activités et de leurs données. L'entreprise a son siège dans le
quartier de La Défense, en région parisienne et compte plus de 2500
collaborateurs dont plus de 250 chercheurs et analystes à travers le monde,
répartis dans 160 pays~\cite{ocd}.

\acrlong{ocd} a été fondée en 2014 et dirigé jusqu'en avril 2021 par Michel Van
Den Berghe. La marque est créée en 2016, regroupant les activités en
cybersécurité d'Orange Consulting et d'Atheos, cabinet de conseil racheté en
2014~\cite{rachat-atheos}.  La filiale a ensuite évoluée avec le rachat de Lexsi
en 2016 puis SecureData et SecureLink en 2019 pour une valeur de 515 millions
d'euros~\cite{rachat-securelink}. La direction est reprise en avril 2021 par
Hugues Foulon.

Financièrement et humainement, l'entreprise a connu une forte croissance depuis
sa création. Entre 2016 et 2020, \gls{ocd} a eu une croissance de chiffre
d'affaires de 21,3\% en moyenne, pour atteindre 300 millions d'euros en 2020 et
768 millions d'euros en 2020 aussi en prenant en compte toutes les acquisitions
de la filiale~\cite{ocd}. Le résultat net quant à lui a quelque peu oscillé au
fur et à mesure des années, mais était par exemple de 2,984 millions d'euros en
2016, 883 milliers d'euros en 2018 et 1,826 millions d'euros en 2020. L'effectif
salarial est lui aussi en augmentation assez régulière depuis 2015 avec en
moyenne 30,0\% de croissance jusqu'à fin 2020, pour atteindre plus de 1100
employés début 2021~\cite{finances-ocd}. L'ensemble du détail des finances de
l'entreprise est mis à disposition en annexe~\ref{sec:annexes::finances-ocd}.

\acrlong{ocd} propose une notable variété de services informatiques en sécurité.
Ils contiennent en particulier au niveau purement technique: du conseil, de
l'audit organisationnel et de sécurité par tests d'intrusion, de la détection de
menaces de manière transparente aux clients et des réponses automatisées, de la
conception d'architectures nouvelles ou en vue d'une migration avec de
l'\gls{integ}, et enfin du support et du \gls{soc} pour atteindre un \gls{mcs}.
Il y a aussi des services plus orienté sur l'axe humain de la sécurité: des
formations visant à sensibiliser les employées d'une entreprise cliente, mais
aussi de l'assistance à la gestion de crise~\cite{ocd}.  Puisque je n'ai fait
partie que de l'équipe d'\gls{integ}, je ne serais en mesure de préciser la
structure interne et le fonctionnement précis des autres équipes. Leur
intervention dans le déroulé d'un projet complet sera cependant abordé en de
plus amples détails dans la partie~\ref{cha:mission} --- \nameref{cha:mission}.

En particulier, l'offre de filtrage délocalisé et réponse automatisée à incident
est réalisée grâce à des infrastructures que l'entreprise a conçues et maintient
soi-même. \acrlong{ocd} héberge au sein de 26 centres de détection dans 13 pays
du monde de quoi analyser en moyenne 50 milliards d'évènements par jour, ce qui
permet notamment de révéler et rapidement clotûrer en moyenne plus de 200 sites
Web détectés malveillants par jour.


\section{Acteurs du marché}%
\label{sec:intro::acteurs}

La sécurité informatique est un enjeu majeur pour toute entreprise et surtout
constant dans le temps. En effet, la migration vers une informatique généralisée
rend la surface d'attaque en général de plus en plus importante. Les attaques
informatiques sont réalisées en permanence et sont surtout de types très variés.

Rien que dans les ménages français par exemple, les personnes faisant face à des
problèmes de sécurité rencontrent huit catégories d'attaques différentes et
déclarent dans 42,5\% des cas avoir été sujet à du hameçonnage par réception
d'un message invitant à se connecter à un site Web frauduleux et dans 21,4\% des
cas par redirection vers un site frauduleux invitant à fournir des informations
personnelles lors d'une navigation Web~\cite{attack-types}.

Cela concerne tout aussi bien les entreprises du secteur privés: 16\% des
sociétés de 10 personnes ou plus implantées en France déclarent avoir vécu un
incident de sécurité informatique en 2018. Les sociétés de 250 personnes ou plus
sont deux fois plus touchées~\cite{companies-security}. Le simple coût lié à la
gestion de la crise déclenchée par une attaque et aux réparations qui
s'ensuivent est conséquent: il est aux États-Unis en moyenne de 27,4 millions de
dollars et en France de 9,72 millions de dollars USD~\cite{attack-costs}. Cela
les amène alors à investir un budget et un temps considérable pour se protéger
en conséquence: en 2019, 87\% de sociétés de 10 personnes ou plus réalisent des
activités en lien avec la sécurité de leur système d'information. Par ailleurs,
67\% des sociétés ont recours à des prestataires pour réaliser des activités de
sécurité informatique, parfois en plus des employés de l'entreprise, et 20\% les
font réaliser uniquement par leurs propres employés, tandis que les grandes
entreprises uniquement le font à 31\%. Ces activités comprennent notamment de
l'édition de communications internes: 26\% des sociétés ont une documentation
sur les mesures, pratiques ou procédures en matière de sécurité des systèmes
d'information, autant qu'en 2015. C'est le cas de 71\% des sociétés de 250
personnes ou plus. Quelle que soit leur taille, sept sociétés sur dix ont défini
ou révisé cette documentation au cours de l'année
écoulée~\cite{companies-security}.

Cette masse d'entreprises ayant recours à des prestataires de services externes
pour les aider à réaliser toutes ces activités particulièrement chronophages est
alors très importante. Le marché associé est donc évidemment conséquent: en
2020, il a représenté à l'échelle globale environ 133 milliards de dollars
USD~\cite{security-market}, dont 96,3 en services de
consultation~\cite{security-consulting-market}, ce qui inclus la grande partie
des activités d'\gls{ocd} par exemple. \acrlong{ocd} est en plus de cela sur la
plupart des fronts de la sécurité (voir la section
précédente~\ref{sec:intro::entreprise}), puisque sert notamment à plus de 8000
clients de tous secteurs d'activité~\cite{ocd}. Son marché direct est alors de
très grande taille et les acteurs y participant en grand nombre. Quant à
certains en particulier, nous citerons par exemple Atos et Wavestone pour la
partie conseil et conception d'architectures, Synacktiv et Quarkslab pour la
partie audit de sécurité et tests d'intrusion, ou encore Login Sécurité en
partie pour les opérations de \gls{soc}.


\chapter{Mission}%
\label{cha:mission}

Ma mission était, d'un point de vue assez global, de participer aux divers
projets et activités de l'équipe d'ingénieurs en \gls{integ} faisant partie du
département <<Solutions de confiance>> d'\acrlong{ocd}. Cette équipe est
elle-même composée de deux sous-équipes: le \gls{ocd-build} et le \gls{ocd-run}.
Le \gls{ocd-build} est chargé d'une partie de la conception de l'architecture à
mettre en place pour répondre aux besoins d'un client, puis surtout de son
implémentation effective qui nécessite souvent une intervention matérielle et
l'établissement de configurations logicielles cibles, pour enfin réaliser une
migration vers le nouveau système. Le \gls{ocd-run} quant à lui est chargé
d'assurer le bon fonctionnement du système implémenté par le \gls{ocd-build} de
sorte qu'il soit opérationnel la majeure partie du temps, de surveiller la
sécurité du parc informatique en question et donc réaliser le rôle de \gls{soc},
mais aussi de faire évoluer les configurations logicielles au fur et à mesure
des demandes du client par voie de tickets. J'ai, pour ma part, rejoint en
particulier l'équipe \gls{ocd-build} pour la durée entière de ce stage.


\section{Déroulé d'un projet}%
\label{sec:mission::deroule-projet}

Ces deux sous-équipes plus d'autres d'\gls{ocd} travaillent ensemble pour
réaliser de nombreux projets. Le déroulé typique d'un projet est le suivant,
principalement orienté du point de vue du \gls{ocd-build}:

\begin{itemize}

    \item Optionnellement, les équipes d'audit et de conseil ont révélé des
        problèmes de sécurité que l'entreprise ayant pris contact avec \gls{ocd}
        ne juge pas acceptable face aux risque potentiellement encourus par
        rapport à la criticité des applications, systèmes et informations
        concernés. La partie conseil va en particulier intervenir surtout pour
        la gestion des opérations suite à la mise en évidence des faiblesses en
        question, pour définir des objectifs les plus clairs possibles en terme
        de sécurité et pour établir une stratégie d'évolution pour les atteindre
        en temps souhaité.

    \item Un projet peut tout aussi bien commencer à l'inverse par une prise de
        contact du client vers \gls{ocd}, par exemple suite à un appel d'offres,
        avec des objectifs déjà proprement définis et pour aller directement
        vers l'implémentation. Il s'agit dans ce cas en général de construction
        d'infrastructures entièrement nouvelles ou d'une migration relativement
        importante de systèmes pour laquelle le client a été en mesure d'en
        identifier la nécessité. Une partie non-négligeable de la conception est
        alors réalisée par les \gls{presales} en collaboration avec le client.
        Leur rôle va principalement être de comprendre et synthétiser les
        besoins du client pour aller vers un premier jet d'une architecture
        adaptée. Si la taille du parc informatique concerné est particulièrement
        conséquente, un ou plusieurs architectes de systèmes peuvent aussi
        intervenir.

    \item Une fois la majeure partie de l'architecture conçue, le
        \gls{ocd-build} commence alors à participer au projet pour tendre vers
        une spécification précise des changements qui vont être réalisés ou de
        l'infrastructure qui va être effectivement implémentée. Ceci commence en
        général par une \gls{rli}. C'est une réunion de coordination interne à
        \gls{ocd} en début de projet pour réaliser une transition entre les
        \gls{presales} et les équipes de \gls{ocd-build} afin de transmettre les
        informations importantes. Elle est en général relativement courte car il
        s'agit principalement de se mettre d'accord sur ce qui va être présenté
        au client et ainsi de préparer la \gls{rle}. Cette dernière est une
        réunion de début de projet pour le \gls{ocd-build} suite à la \gls{rli}
        cette fois-ci avec le client présent pour d'abord faire rencontrer les
        différentes personnes des deux côtés, puis commencer à préparer une
        spécification précise de l'implémentation à réaliser et le continuer
        selon la taille du projet dans les jours, semaines ou mois suivants lors
        d'une série d'ateliers techniques.

    \item Le \gls{ocd-build} ayant réellement commencé le projet, ils vont
        chercher à rédiger une spécification de l'architecture à implémenter et
        déployer afin d'obtenir un accord de la part du client à son sujet et
        donc ainsi d'avoir à portée de main une forme de guide contenant toutes
        les informations à configurer sur les différents composants du système
        global à mettre ne place. Cela permet aussi de proposer un cadre pour
        les changements en cours de route: si le client avait bien approuvé la
        spécification effective et qu'\acrlong{ocd} l'implémente en entier ou
        presque mais que le client exprime finalement un mécontetement qui
        demanderait de revoir une grande partie de l'architecture, \gls{ocd}
        peut alors s'en défendre en rappelant l'approbation en question; le
        client devra, s'il le souhaite effectivement, renégocier les contrats et
        acheter de nouveau du temps de travail. Selon la taille du projet, cette
        spécification peut prendre plusieurs formes différentes:

    \begin{itemize}

        \item Si le projet est particulièrement conséquent car fait intervenir
            de nombreuses équipes, demande de déployer beaucoup de sites,
            nécessite un grande variété de configurations, types d'équipements
            ou logiciels, \ldots{} il est alors fréquent de séparer cette
            rédaction en deux parties: un \gls{hld} puis un \gls{lld}. Un
            \gls{hld} est un dossier regroupant les spécifications de
            l'architecture à implémenter d'un point de vue assez global en
            partant de l'idée générale de départ proposée par les
            \gls{presales}. Il ne s'agit pas de préciser toutes les valeurs
            effectives des options principales de configurations cibles à mettre
            en place: c'est plus le rôle du \gls{lld} qui en découle. Ce dernier
            est un dossier regroupant les spécifications les plus détaillées
            possibles de l'architecture à implémenter telles que modèles de
            machine à déployer, versions de systèmes d'exploitation à installer,
            interfaces, adresses, routes, options, autorisations, règles,
            objets, \ldots{} à configurer. Il est en général une continuation
            directe du \gls{hld} associé. Cette séparation en deux permet
            notamment d'un peu mieux gérer la communication car se fait à
            propose des bons sujets lors des ateliers techniques, mais aussi la
            répartition du temps de travail des ingénieurs car évite si possible
            de nombreux aller-retours entre la spécification et
            l'implémentation.

        \item Si le projet n'est à l'inverse pas trop gros, il est souvent
            envisagé de passer directement à un \gls{dsd}. C'est un dossier très
            similaire à un \gls{lld}, car comportera le même genre de contenu
            précis, mais est plutôt employé lorsque la taille du projet n'est
            pas trop conséquente et permet donc de rédiger directement les
            spécifications de l'implémentation en un temps moins long puisque la
            plupart des sujets abordés lors d'ateliers techniques avec le client
            ne posent pas de problème particulier et ne sont donc pas bloquants.
            Pour résumer, un \gls{dsd} est une forme de fusion entre un
            \gls{hld} et un \gls{lld}.

    \end{itemize}

    \item Une fois la spécification rédigée, ajustée et approuvée par le client,
        l'équipe d'ingénieurs \gls{ocd-build} dédiée au projet commence
        l'implémentation. Cela démarre régulièrement par une étape de
        \gls{staging}: placer les équipements dans un environnement de test qui
        s'approche le plus possible de l'environnement de production effectif.
        L'intérêt est que cela permet de vérifier certaines parties du système,
        comme par exemple l'interaction avec une base de données externe, qui
        sont dans un environnement de test plutôt en local sur le même hôte,
        mais aussi d'être en mesure d'avoir un contrôle plus important de
        l'environnement global, comme par exemple vider la base de données et y
        charger des données de test qui couvre des cas précis. Cela permet
        aussi, lorsque cela peut s'appliquer, de vérifier le bon fonctionnement
        de machines physiques pour gérer en interne les potentiels dégâts liés
        au transport des marchandises. Avec une première configuration embarquée
        dans les équipements, le déploiement dans l'environnement de production
        peut donc se faire de manière plus rapide sur place car il s'agit
        principalement de visser les machines, de les alimenter, connecter et
        démarrer, mais aussi de manière plus sereine car une bonne partie des
        cas ont déjà été vérifiés. Après cette étape, l'architecture est
        déployée et configurée petit à petit jusqu'à atteindre un état cible ou
        presque.

    \item Quand l'implémentation arrive vers sa fin, le \gls{ocd-build} commence
        à rédiger en parallèle un \gls{dtv} par site déployé. Il regroupe un
        ensemble de fiches de tests d'acceptation à effectuer avec succès pour
        considérer le déploiement du site et la migration globale comme des
        succès, et ainsi que le site soit considéré comme livré au client. Ces
        tests peuvent très bien être entièrement automatisés ou bien faire
        intervenir une certaine part d'actions manuelles: c'est en général le
        cas pour les vérifications de bascule lorsque que la redondance physique
        fait partie de l'architecture globale.  Les différents \glspl{dtv} sont
        complétés et validés par le client au fur et à mesure de quelques
        itérations, comme la majeure partie des documents produits.

    \item Après que le \gls{dtv} d'un site est bien validé et que son
        implémentation est terminée par le \gls{ocd-build}, une \gls{vabf} est
        planifiée et effectuée avec le client. Il s'agit d'une session lors de
        laquelle tous les tests d'acceptation du \gls{dtv} associé sont exécutés
        et leur résultats respectifs rapportés dans le document. Si un nombre
        particulièrement important de tests échouent, il est tout à fait
        envisageable de réaliser une nouvelle itération d'implémentation et par
        la suite de planifier et exécuter une seconde \gls{vabf} entière. Si
        seulement un nombre relativement faible de tests échouent, que certains
        n'ont simplement pas pu être réalisés puisque des éléments humains ou
        techniques manquaient ou que certains sont considérés non statisfaisants
        --- c'est-à-dire une forme d'échec partiel où dans certaines conditions
        estimées non courantes les systèmes ne fonctionnent pas comme attendu,
        il est alors possible de considérer la \gls{vabf} comme un succès avec
        quelques retenues tout de même. Les problèmes sont réglés dans les jours
        qui suivent par le \gls{ocd-build} et vérifiés en dehors d'une seconde
        \gls{vabf}.

    \item Si les résultats de tous les tests sont des succès, la \gls{vabf} est
        bien complétée, les ingénieurs \gls{ocd-build} sont libérés du projet et
        le site en question entre dans une période dite de \gls{vsr}. Pendant
        cette période d'une durée définie en accord avec le client, celui-ci
        utilise l'environnement de production cible déployé et fonctionnel du
        point de vue des tests déjà effectués, mais cette fois-ci pour vérifier
        que cela reste le cas dans la durée, c'est-à-dire qu'il n'y ait pas de
        comportement erratique ou d'instabilité seulement à certains moments de
        la journée par exemple, ou lorsque les systèmes sont soumis à de fortes
        charges, ou encore de temps en temps de manière totalement stochastique.
        Cette durée est en général de deux semaines pour les projets
        d'\acrlong{ocd}, mais peut être étendue principalement sur demande du
        client pour des projets dont ce genre d'aspects est jugé
        particulièrement critique au bon fonctionnement global du système
        considéré. Lors de la \gls{vsr}, l'équipe de \gls{ocd-build} réalise les
        corrections demandées par le client en mode <<meilleur effort>>,
        c'est-à-dire que les ingénieurs en question vont y passer le temps
        qu'ils peuvent, mais n'en feront pas pour autant une priorité puisqu'ils
        ne sont plus affectés au projet concerné et peuvent très bien faire déjà
        partie d'un autre projet. Cependant, si un dysfonctionnement majeur se
        produit et qu'il s'agit d'une erreur de la part de l'équipe, \gls{ocd}
        se doit de refaire une itération de \gls{ocd-build}, mais s'il s'agit
        d'un problème qui ne leur est pas attribuable, le client devra alors
        payer pour une nouvelle itération s'il le souhaite.

    \item Optionnellement, un \gls{dex} est rédigé en parralèle et sur demande
        du client. C'est une compilation systématique des procédures et des
        opérations que l'administrateur ou l'opérateur du système effectue. Un
        tel dossier leur sert aussi souvent de référence, autant pour du partage
        d'informations que de base pour leur travail quotidien. Il peut
        également contenir des descriptions pour le traitement de demandes
        spéciales et d'éventualités, mais aussi de petits changements à réaliser
        eux-mêmes. Cela permet notamment de clarifier, documenter l'architecture
        effectivement implémentée \textit{versus} celle initialement prévue dans
        un éventuel \gls{dsd}. Ce genre de document est en général produit
        lorsque le client va être en mesure d'administrer son propre système
        potentiellement sans aucune intervention de la part du \gls{ocd-build},
        et pour ainsi lui fournir l'ensemble des éléments pratiques nécessaires
        de sorte que toutes ses opérations régulières puissent être réalisées.

    \item Quand toutes ces étapes-là sont bien complétées, le projet est en
        quelque sorte terminé, mais l'interaction entre \gls{ocd} et le client
        continue cependant par une transition vers le \gls{ocd-run}. Après une
        passation des informations importantes des équipes du \gls{ocd-build}
        vers le \gls{ocd-run}, celui-ci prend pleinement possession des
        interventions à réaliser sur le parc informatique précédemment déployé
        et ainsi assurer le rôle de \gls{soc} si cela fait partie du contrat
        négocié avec le client. Quelconque modification que souhaite faire le
        client, si encore une fois ils n'ont pas exprimé et négocié dans le
        contrat le besoin d'aussi pouvoir gérer les systèmes eux-même, il devra
        alors ouvrir un ticket auprès du \gls{ocd-build} afin que ces derniers
        appliquent le changement souhaité. Un tarif au tiquet ou groupe de
        tiquets est appliqué.

\end{itemize}

En somme, ce mode de travail est assez similaire à la méthode dite de la
<<cascade>> où les étapes sont enchaînées sans vraiment prendre en compte la
possibilité de retour en arrière et encore moins de cycle, ou en tout cas ne pas
la souhaiter. Même si elle a en théorie certains avantages, puisque cherche à
formaliser les choses à l'avance afin de simplement suivre le flot <<naturel>>
des évènements, elle a tout de même certaines absurdités. En effet, si on
cherche à la respecter de la manière la plus stricte possible et que, par
exemple, on se rende compte d'une erreur dans la spécification lors de
l'implémentation voire des tests, que fait-on? on continue la chaîne en
outre-passant le problème? on abandonne le projet? on démissionne? Même si on
relâche un peu la définition, cela n'est en fait pas très adapté à certains
projets, comme nous le verrons plus tard. Évidemment, une solution est de
changer de méthode en appliquant par exemple le <<cycle en V>>, au moins de
manière partielle, mais selon mon expérience chez \gls{ocd}, je n'ai pas
l'impression qu'il s'agisse vraiment de ce qui est employé. Cependant, je pense
aussi qu'il n'est pas non plus idéal d'aller à l'extrême opposé, c'est-à-dire de
considérer que rien n'est prévisible et de s'appuyer sur un cycle fréquent pour
effacer les bosses soulevées par des problèmes de conception qui en réalité
auraient pu être prévus. Un minimum de conception à l'avance est toujours
possible et même plus que souhaitable.


\section{Objectifs}%
\label{sec:mission::objectifs}

\acrlong{ocd} étant principalement un prestataire de service, une bonne partie
des objectifs du stage n'ont pas pu être définis à l'avance et de manière
précise, puisque les activités que j'y ai menées dépendaient fortement des
clients et de leur projets en cours. On peut tout de même mettre en évidence les
points suivants:

\begin{itemize}
    \item Apprendre en auto-formation à utiliser les divers produits faisant
        partie des spécialités \gls{ocd} et pouvant être concernés par un de ses
        projets pour aller vers une maîtrise et ainsi participer aux projets en
        tant qu'ingénieur.
    \item Concevoir et implémenter des architectures sécurisées.
    \item Élaborer des documentations techniques, spécifications générales et
        détaillées.
    \item Réaliser des projets: ingénierie, déploiement, migration / mise en
        production, transferts de connaissances, \ldots
    \item Contribuer à des activités purement internes telles qu'améliorer les
        infrastructures des salles de test, y déployer de nouvelles maquettes,
        participer au développement de nouvelles expertises, \ldots
\end{itemize}


\section{Premières missions}%
\label{sec:mission::prems}

\subsection{Auto-formation}%
\label{sub:mission::prems::auto-formation}

\subsection{Documentation de la salle de tests}%
\label{sub:mission::prems::doc-salle-tests}


\section{Relations humaines et management}%
\label{sec:mission::rhm}


\chapter{Responsabilité sociétale des entreprises}%
\label{cha:rse}

\section{Economie}%
\label{sec:rse::eco}

\section{Social}%
\label{sec:rse::social}

\section{Environnement}%
\label{sec:rse::env}

\section{Gouvernance}%
\label{sec:rse::gouv}

\section{Santé et Sécurité au Travail}%
\label{sec:rse::sst}


\chapter{Bilan}%
\label{cha:bilan}

\section{Retour d'expérience}%
\label{sec:bilan::ret-exp}


\chapter{Conclusion}%
\label{cha:conclu}

\section{Perspectives}%
\label{sec:conclu::persp}


\chapter{Annexes}%
\label{cha:annexes}

\section{Résumé des finances}%
\label{sec:annexes::finances-ocd}

\begin{longtable}{|l|r|r|r|r|}
    \hline
    \textbf{Performance} & \textbf{2020} & \textbf{2019} & \textbf{2018} &
    \textbf{2017} \\ \hline
        Chiffre d'affaires (\texteuro) & 301M & 274M & 219M & 191M \\ \hline
        Marge brute (\texteuro) & 204M & 179M & 153M & 126M \\ \hline
        EBITDA --- EBE (\texteuro) & 12,5M & 13,3M & 8,97M & 5,94M \\ \hline
        Résultat d'exploitation (\texteuro)
            & 3,63M & 4,8M & 1,43M & 1,2M \\ \hline
        Résultat net (\texteuro) & 1,83M & 2,13M & 883K & 1,39M \\ \hline
    \textbf{Croissance} & \textbf{2020} & \textbf{2019} & \textbf{2018} &
    \textbf{2017} \\ \hline
        Taux de croissance du CA (\%) & 9,8 & 25,1 & 14,8 & 35,6 \\ \hline
        Taux de marge brute (\%) & 67,7 & 65,2 & 69,7 & 65,8 \\ \hline
        Taux de marge d'EBITDA (\%) & 4,2 & 4,8 & 4,1 & 3,1 \\ \hline
        Taux de marge opérationnelle (\%) & 1,2 & 1,8 & 0,7 & 0,6 \\ \hline
    \textbf{Gestion BFR} & \textbf{2020} & \textbf{2019} & \textbf{2018} &
    \textbf{2017} \\ \hline
        BFR (\texteuro) & 38,4M & 40,8M & 2,07M & 7,33M \\ \hline
        BFR exploitation (\texteuro) & 55,8M & 69,9M & 55,6M & 58,2M \\ \hline
        BFR hors exploitation (\texteuro) & -17,4M
            & -29,1M & -53,5M & -50,9M \\ \hline
        BFR (j de CA) & 46,6 & 54,3 & 3,4 & 14 \\ \hline
        BFR exploitation (j de CA) & 67,7 & 93,1 & 92,6 & 111 \\ \hline
        BFR hors exploitation (j de CA) & -21,1 & -38,8 & -89,2 & -97,4 \\ \hline
        Délai de paiement clients (j) & 168 & 194 & 176 & 199 \\ \hline
        Délai de paiement fournisseurs (j) & 173 & 175 & 140 & 142 \\ \hline
        Ratio des stocks / CA (j) & 12,9 & 16,7 & 10,9 & 7,7 \\ \hline
    \textbf{Autonomie financière} & \textbf{2020} & \textbf{2019} &
    \textbf{2018} & \textbf{2017} \\ \hline
        Capacité d'autofinancement (\texteuro)
            & 10,8M & 10,6M & 8,27M & 5,28M \\ \hline
        Capacité d'autofinancement / CA (\%) & 3,6 & 3,9 & 3,8 & 2,8 \\ \hline
        Fonds de roulement net global (\texteuro)
            & 41,2M & 43M & 6,02M & 10,9M \\ \hline
        Couverture du BFR & 1,1 & 1,1 & 2,9 & 1,5 \\ \hline
        Trésorerie (\texteuro) & 2,96M & 2,31M & 3,98M & 3,48M \\ \hline
        Dettes financières (\texteuro) & 40M & 50,8M & 16M & 21M \\ \hline
        Capacité de remboursement & 3,4 & 4,6 & 1,5 & 3,3 \\ \hline
        Ratio d'endettement (Gearing) & 0,5 & 0,7 & 0,2 & 0,3 \\ \hline
        Autonomie financière (\%) & 22,9 & 22,8 & 29,5 & 30,2 \\ \hline
        Taux de levier (DFN/EBITDA) & 3 & 3,7 & 1,3 & 3 \\ \hline
    \textbf{Solvabilité} & \textbf{2020} & \textbf{2019} & \textbf{2018} &
    \textbf{2017} \\ \hline
        Etat des dettes à 1 an au plus (\texteuro)
            & 199M & 182M & 142M & 131K \\ \hline
        Liquidité générale & 1,2 & 1,2 & 1 & 1,04K \\ \hline
        Couverture des dettes & 2,3 & 1,9 & 7,1 & 4,6 \\ \hline
    \textbf{Rentabilité} & \textbf{2020} & \textbf{2019} & \textbf{2018} &
    \textbf{2017} \\ \hline
        Marge nette (\%) & 0,6 & 0,8 & 0,4 & 0,7 \\ \hline
        Rentabilité sur fonds propres (\%) & 2,6 & 3,2 & 1,3 & 2,1 \\ \hline
        Rentabilité économique (\%) & 0,6 & 0,7 & 0,4 & 0,6 \\ \hline
        Valeur ajoutée (\texteuro) & 103M & 94,2M & 77M & 62,6M \\ \hline
        Valeur ajoutée / CA (\%) & 34,1 & 34,4 & 35,1 & 32,8 \\ \hline
    \textbf{Structure d'activité} & \textbf{2020} & \textbf{2019} &
    \textbf{2018} & \textbf{2017} \\ \hline
        Salaires et charges sociales (\texteuro)
            & 89,2M & 78,9M & 66,6M & 55,3M \\ \hline
        Salaires / CA (\%) & 29,6 & 28,8 & 30,4 & 29 \\ \hline
        Impôts et taxes (\texteuro) & 4,11M & 4,09M & 3,27M & 2,66M \\ \hline
    \caption{Résultats financiers~\cite{finances-ocd}}%
    \label{tab:annexes::finances-ocd::tab}
\end{longtable}


\printbibliography[title = Références]

\listoffigures

\glsaddall{}
\printglossary[type = main]
\printglossary[type = \acronymtype, title = Acronymes]



\end{document}
