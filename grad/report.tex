\documentclass[12pt, oneside, a4paper, titlepage]{report}
\usepackage{url}
\usepackage[hidelinks]{hyperref}
\usepackage{csquotes}
\usepackage{fancyvrb}
\usepackage[toc,acronym]{glossaries}
\usepackage[export]{adjustbox}
\usepackage{booktabs}
\usepackage{longtable}
\usepackage{tabularx}
\usepackage{pbox}
\usepackage[main=french]{babel}
\usepackage{moresize}
\usepackage{xcolor}
\usepackage{soulutf8}
\usepackage[T1]{fontenc}
\usepackage{lmodern}
\usepackage{graphicx}
\usepackage{calc}
\usepackage{geometry}

\geometry{%
    paper=a4paper,  % Paper size, change to letterpaper for US letter size.
    top=3cm,  % Top margin.
    bottom=3cm,  % Bottom margin.
    left=3.5cm,  % Left margin.
    right=3.5cm,  % Right margin.
    headheight=0.75cm,  % Header height.
    footskip=1cm,  % Space from the bottom margin to the baseline of the footer.
    headsep=0.5cm,  % Space from the top margin to the baseline of the header.
    %showframe,  % Uncomment to show how the type block is set on the page.
}

\usepackage[sorting=none]{biblatex}
\addbibresource{refs.bib}
\setacronymstyle{long-short-desc}
\loadglsentries{glossary.tex}
\makeglossaries{}

\newcommand{\nomPrenom}{{%
    \fontsize{14}{14}\selectfont Paul Mabileau
}}
\newcommand{\nomEntreprise}{{%
    \fontsize{16}{16}\selectfont ORANGE CYBERDÉFENSE
}}

\newcommand{\nomAdresseEntreprise}{{%
    \fontsize{12}{12}\selectfont Orange Cyberdéfense, 2 rue Christophe Colomb,
    91300 Massy.
}}

\newcommand{\titreMission}{{%
    \fontsize{14}{14}\selectfont Intégration sécurité: déploiement d'une
    architecture SD-WAN pour trois sites de production.
}}

\newcommand{\nomDirecteurStage}{{%
    \fontsize{12}{12}\selectfont Valentin Goron
}}

\newcommand{\nomConseillerStage}{{%
    \fontsize{12}{12}\selectfont Olivier Paul
}}

\newcommand{\anneeUniversitaire}{{%
    \fontsize{15}{15}\selectfont 2020/2021
}}

\newcommand{\dateDebut}{{%
    \fontsize{12}{12}\selectfont 22/02/2021
}}

\newcommand{\dateFin}{{%
    \fontsize{12}{12}\selectfont 20/08/2021
}}

\newcommand{\vap}{{%
    \fontsize{11}{11}\selectfont SSR
}}



\begin{document}

\begin{titlepage}
    \newpage
\thispagestyle{empty}


\begingroup
    \begin{tabular}{ll}
        % LOGO DE TELECOM SUDPARIS
        \begin{minipage}[c]{0.0\textwidth}
            \begin{flushleft}
                \includegraphics[width=3cm]{img/tsp.png}
            \end{flushleft}
        \end{minipage}
        &
        % ANNEE UNIVERSITAIRE
        \hspace{-3em}
        \begin{minipage}[c]{1.0\textwidth}
            \begin{flushright}
                \textbf{ANNÉE \anneeUniversitaire}
            \end{flushright}
        \end{minipage}
        \hspace{\stretch{1}}
    \end{tabular}
    \vskip 4.0em
\endgroup

\begingroup
    \begin{center}
        \Large{\textbf{RAPPORT de <<MISSION en ENTREPRISE>>}}
        \vskip 2.0em

        présenté par:
        \vskip 2.0em
        \textbf{\nomPrenom}

        \vskip 3.0em
        Mission effectuée de \dateDebut{} au \dateFin{} chez
        \vskip 2.0em
        \textbf{\nomEntreprise}
        \vskip 3.5em

        Sujet de la mission:
    \end{center}

     % CADRE POUR LE TITRE DE LA MISSION
    \parbox[b]{1.0\textwidth-1cm}{%
        \hrule height 1.5pt
        \vrule width 1.5pt
        \hspace{1.80cm}
        \parbox[b]{\textwidth-5cm}{%
            \vskip 0.6em
            \center\large{\textbf{\titreMission}}\endcenter{}
            \vskip 0.6em
        }
        \hspace{1.80cm}
        \vrule width 1.5pt
        \hrule height 1.5pt
    }
    \vskip 3.5em
\endgroup

\begingroup
    Directeur de stage: \textbf{\nomDirecteurStage}
    \vskip 1.0em
    Conseiller de Stage: \textbf{\nomConseillerStage{}}
\endgroup

\vskip 4.5em

\begingroup
    \hrule height 1.0pt
    \vskip 0.5em
    \noindent Travail effectué pour la société: \nomAdresseEntreprise{}
\endgroup


\newpage

\end{titlepage}

\tableofcontents

\newpage
\thispagestyle{empty}


\begingroup
    \begin{tabular}{ll}
    % LOGO DE TELECOM SUDPARIS
        \begin{minipage}[c]{0.0\textwidth}
            \begin{flushleft}
                \includegraphics[width=4cm]{img/tsp.png}
            \end{flushleft}
        \end{minipage}
        &
        % ANNEE UNIVERSITAIRE
        \hspace{-3em}
        \begin{minipage}[c]{1.0\textwidth}
            \begin{flushright}
                \Large{\textbf{Stage de 3\up{ème} année}}\\
                \Large{\textbf{\anneeUniversitaire}}
            \end{flushright}
        \end{minipage}
        \hspace{\stretch{1}}
    \end{tabular}
    \vskip 1.0em
\endgroup

\begingroup
    \noindent \textbf{Nom de l'étudiant}: \nomPrenom{}
    \vskip 0.5em
    \noindent \textbf{VAP}: \vap{}
    \vskip 0.5em
    \noindent \textbf{Entreprise}: \nomEntreprise{}
    \vskip 0.5em
    \noindent \textbf{Dates}: du \dateDebut{} au \dateFin{}
    \vskip 0.5em
    \noindent \textbf{Sujet}: \titreMission{}
    \vskip 3em

    % FR
    \noindent
    \fbox{%
        \parbox{\textwidth}{%
            \vskip 0.8em
            \hspace{0.5cm}
            \vspace{0.4em}
            \large{\textbf{RÉSUMÉ:}}
            \vskip 0.4em
            \vskip 0.8em
        }
    }

    % EN
    \noindent
    \fbox{%
        \centering
        \parbox{\textwidth}{%
            \vskip 0.8em
            \vspace{0.4em}
            \hspace{0.5cm}
            \large{\textbf{ABSTRACT:}}
            \vskip 0.4em
            \vskip 0.8em
        }
    }
    \endgroup


\newpage


\chapter{Remerciements}%
\label{cha:ackn}

Je tiens tout d'abord à remercier mon directeur de stage, Valentin Goron, pour
son suivi régulier du stage et des activités que j'y menais, pour souvent
m'orienter vers une personne prête à m'aider à résoudre quelconque problème. Je
remercie donc Guerric Lemoigne pour son accueil efficace et toutes les petites
opérations et questions qui viennent avec un début de stage, Andrès Marino pour
les dépannages en salle serveur, Arnaud Fauvel pour la rédaction du DEX, Deroi
Tieffi pour avoir été un bon taxi, Geoffrey Giganti pour le coup de pouce pour
démarrer en Stormshield, et j'en oublie d'autres.

Je remercie les chefs de projet avec qui j'ai eu l'occasion de collaborer, ne
serait-ce qu'un peu: Arnaud Creuser et surtout Laurent Saunier pour avoir
brillament géré un projet en mouvement constant, bourrés de changements de
dernière minute et fréquemment bloqué ou retardé par des imprévus sortant de nos
responsabilités.

Merci aussi à Gauthier pour les longues journées de nos débuts de stage passées
en salle de test à documenter, nettoyer, réorganiser et étiqueter tout ce qui
bouge --- ou en l'occurence plutôt pas trop.

Enfin, je tiens tout particulièrement à remercier Sylvain Bardy pour un travail
en binôme qu'il rend toujours efficace mais surtout agréable du fait de son
soutien constant et de sa bonne humeur inconditionnelle. Même lorsque contraint
à devoir jongler entre plusieurs projets dont d'autres bien plus prenant, il
était toujours prêt à répondre à mes nombreuses questions et interrogations,
voire à se libérer d'une minute à l'autre pour résoudre divers problèmes
ensemble sans pour autant le demander. Je pense sincèrement que la réussite du
projet en question et de la fin de mon stage peut lui être attribuée en grande
partie. Merci beaucoup.

\chapter{Introduction}%
\label{cha:intro}

\section{Stage}%
\label{sec:intro-stage}

Aujourd'hui, la sécurité est souvent citée comme un enjeu majeur pour les
entreprises ainsi que pour l'ensemble des acteurs qui l'entourent. Elle n'est
plus confinée uniquement au rôle de l'informaticien. Le domaine devient de plus
en plus important à cause de la transition progressive vers des systèmes
informatiques~\cite{reliance}, de l'Internet et de protocoles réseau sans fil
tel que Bluetooth ou Wi-Fi, et de la constante augmentation du nombre
d'appareils <<intelligents>> tels que téléphones et télévisions modernes et les
nombreux types différents d'appareils constituant <<l'Internet des objets>>. Du
fait de sa complexité grandissante, à la fois en termes politiques et
technologiques, la cybersécurité est aussi un des défis proéminents du monde
contemporain~\cite{global-cyber}. Ces raisons amènent alors de nombreuses
entreprises à investir un temps et un budget significatif à l'établissement
d'une sécurité opérationnelle éstimée adaptée aux contraintes liées à leur cœur
de métier. C'est dans ce cadre que s'incrit l'activité d'Orange Cyberdéfense et
en particulier mon stage.

Mon stage avait pour domaine principal l'intégration d'outils de sécurité pour
le compte de clients d'Orange Cyberdéfense. Comme ceux-ci sont variés, les
projets associés abordent des sujets plus ou moins spécifiques, des enjeux et
contraintes différents, une criticité de la sécurité adaptée, des durées et
budgets variés, \ldots L'idée est donc de les accompagner dans une démarche
d'identification de sujets et problèmes de sécurité les concernant directement,
d'amélioration de leurs infrastructures, logiciels et configurations, et de
conception d'architectures adaptées à l'implémentation de ces changements dans
le cadre d'un projet dédié.

\section{Présentation de l'entreprise}%
\label{sec:intro-entreprise}

\section{Acteurs du marché}%
\label{sec:intro-acteurs}

\chapter{Mission}%
\label{cha:mission}

\section{Objectifs}%
\label{sec:mission-objectifs}

\section{Relations humaines et management}%
\label{sec:mission-rhm}

\chapter{Responsabilité sociétale des entreprises}%
\label{cha:rse}

\section{Economie}%
\label{sec:rse-eco}

\section{Social}%
\label{sec:rse-social}

\section{Environnement}%
\label{sec:rse-env}

\section{Gouvernance}%
\label{sec:rse-gouv}

\section{Santé et Sécurité au Travail}%
\label{sec:rse-sst}

\chapter{Bilan}%
\label{cha:bilan}

\section{Retour d'expérience}%
\label{sec:bilan-retexp}

\chapter{Conclusion}%
\label{cha:conclu}

\section{Perspectives}%
\label{sec:conclu-persp}

\chapter{Annexes}%
\label{cha:annexes}

\printbibliography[title=Références]

\listoffigures

\glsaddall{}
\printglossary[type=main]
\printglossary[type=\acronymtype, title=Acronymes]

\end{document}
