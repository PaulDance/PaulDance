\newpage
\thispagestyle{empty}


\begingroup
    \begin{tabular}{ll}
        % LOGO DE TELECOM SUDPARIS
        \begin{minipage}[c]{0.0\textwidth}
            \begin{flushleft}
                \includegraphics[width=4cm]{img/logo/tsp.png}
            \end{flushleft}
        \end{minipage}
        &
        % ANNEE UNIVERSITAIRE
        \hspace{-3em}
        \begin{minipage}[c]{1.0\textwidth}
            \begin{flushright}
                \Large{\textbf{Stage de 3\up{ème} année}}\\
                \Large{\textbf{\anneeUniversitaire}}
            \end{flushright}
        \end{minipage}
        \hspace{\stretch{1}}
    \end{tabular}
    \vskip 1em
\endgroup

\begingroup
    \noindent \textbf{Nom de l'étudiant}: \nomPrenom{}
    \vskip 0.5em
    \noindent \textbf{VAP}: \vap{}
    \vskip 0.5em
    \noindent \textbf{Entreprise}: \nomEntreprise{}
    \vskip 0.5em
    \noindent \textbf{Dates}: du \dateDebut{} au \dateFin{}
    \vskip 0.5em
    \noindent \textbf{Sujet}: \titreMission{}
    \vskip 2em

    % FR
    \noindent
    \hspace*{-5em}
    \fbox{%
        \centering
        \parbox{\textwidth+8.5em}{%
            \vskip 0.8em
            \hspace{0.5cm}
            \vspace{0.4em}
            \large{\textbf{RÉSUMÉ:}}
            \vskip 0.4em

            {%
                \fontsize{12.5}{12.5}\selectfont

                Ce stage de fin d'études m'a placé dans le rôle d'ingénieur en
                sécurité dans l'équipe d'intégration d'équipements de
                production. Il a d'abord consisté en une montée en compétences
                sur les principaux produits des fabriquants Fortinet et Palo
                Alto en auto-formation.  Ensuite, j'ai participé à la rédaction
                d'un dossier d'exploitation d'une architecture déjà complétée.
                Enfin, j'ai contribué à la réalisation d'un projet entier de
                déploiement d'une architecture SD-WAN reliant entre eux trois
                sites de production d'un client et permettant divers accès aux
                sous-réseaux de chaque site aux partenaires du client et au
                client lui-même. J'ai donc été en mesure de manipuler en
                pratique des technologies de sécurité réseau et applicative
                mettant en jeu divers protocoles dédiés. Cela m'a ainsi mené à
                maîtriser les fonctionnalités centrales, mais aussi certaines
                plus spécifiques, des pares-feux et autres produits Fortinet.

            }

            \vskip 0.8em
        }
    }

    % EN
    \noindent
    \hspace*{-5em}
    \fbox{%
        \centering
        \parbox{\textwidth+8.5em}{%
            \vskip 0.8em
            \vspace{0.4em}
            \hspace{0.5cm}
            \large{\textbf{ABSTRACT:}}
            \vskip 0.4em

            {%
                \fontsize{12.5}{12.5}\selectfont

                This graduation internship put me in the role of a security
                engineer part of the integration team for production appliances.
                It first started with a skill development period by
                self-learning the basics of the Fortinet and Palo Alto
                manufacturers' main products. Then, I participated in the
                writing of an exploitation guide for an already-completed
                project's architecture. Finally, I helped carrying out an entire
                project which consisted in deploying an SD-WAN architecture
                connecting together three production sites of a client and
                allowing the client's partners and the client itself various
                types of access to the several subnets of every production site.
                I was therefore able to directly manipulate network and
                application security technologies implicating a few dedicated
                protocols. It thus lead me to master the core, but also some
                more specific functionalities for the Fortinet firewalls and
                other products.

            }

            \vskip 0.8em
        }
    }
    \endgroup


\newpage
