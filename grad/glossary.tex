%\newglossaryentry{}{%
%    name=,
%    description={},
%}

\newglossaryentry{integ}{%
    name=intégration,
    description={%
        L'intégration de système est une activité d'ingénierie qui consiste en
        le processus de mise en relation de différents systèmes, souvent
        disparates, dans le but qu'ils deviennent composants d'un nouveau
        système plus global. En informatique, il peut s'agir de connexion
        physiques entre machines ou logiques entre logiciels. En sécurité, cela
        ajoute certaines contraintes, notamment de ne pas s'arrêter uniquement
        au bon fonctionnement du système global mais d'inclure aussi le
        non-fonctionnement souhaité de ce qui n'appartient pas au système
    },
}


\newacronym[description={filiale d'Orange en cybersécurité}]
    {ocd}{OCD}{Orange Cyberdéfense}

\newacronym[description={%
    filiale d'Orange proposant des services de télécommunication et
    informatiques pour les entreprises en France et dans le monde
}]{obs}{OBS}{Orange Business Services}

\newacronym[description={%
    l'ensemble de précautions, vérifications régulières, redondances,
    améliorations, \ldots permettant de rester à un niveau de risques estimé
    acceptable selon les critères de l'organisation concernée
}]{mcs}{MCS}{Maintient en Conditions de Sécurité}

\newacronym[description={%
    pôle ou équipe chargée de l'implémentation et l'exécution d'un suivi en
    partie automatisé d'un parc informatique pour être en mesure de détecter et
    de remédier aux menaces le concernant le plus rapidement possible
}]{soc}{SOC}{Security Operation Center}

%\newacronym[description={}]{}{}{}
