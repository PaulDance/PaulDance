%\newglossaryentry{}{%
%    name = ,
%    description = {},
%}

\newglossaryentry{integ}{%
    name = intégration,
    description = {%
        L'intégration de système est une activité d'ingénierie qui consiste en
        le processus de mise en relation de différents systèmes, souvent
        disparates, dans le but qu'ils deviennent composants d'un nouveau
        système plus global. En informatique, il peut s'agir de connexion
        physiques entre machines ou logiques entre logiciels. En sécurité, cela
        ajoute certaines contraintes, notamment de ne pas s'arrêter uniquement
        au bon fonctionnement du système global mais d'inclure aussi le
        non-fonctionnement souhaité de ce qui n'appartient pas au système
    },
}

\newglossaryentry{ocd-build}{%
    name = \textit{Build},
    description = {%
        ou <<construction>>, c'est-à-dire <<implémentation>>~: sous-équipe de
        l'\gls{integ} chargée d'une partie de la conception de l'architecture à
        mettre en place pour répondre aux besoins d'un client, puis surtout de
        son implémentation effective qui nécessite souvent une intervention
        matérielle et l'établissement de configurations logicielles cibles, pour
        enfin réaliser une migration vers le nouveau système
    },
}

\newglossaryentry{ocd-run}{%
    name = \textit{Run},
    description = {%
        ou <<exécution>>, c'est-à-dire <<exploitation>>~: sous-équipe de
        l'\gls{integ} chargée d'assurer le bon fonctionnement du système
        implémenté par le \gls{ocd-build} de sorte qu'il soit opérationnel la
        majeure partie du temps, de surveiller la sécurité du parc informatique
        en question et donc réaliser le rôle de \gls{soc}, mais aussi de faire
        évoluer les configurations logicielles au fur et à mesure des demandes
        du client par voie de tickets
    },
}

\newglossaryentry{presales}{%
    name = avant-ventes,
    description = {%
        ou dans le cas de l'informatique, en particulier la sécurité, aussi
        appelé <<ingénieur avant-vente>> est une personne qui intervient dans le
        procédé de l'acquisition d'un client jusqu'à la vente en elle-même. La
        personne va avoir pour rôle d'entrer en contact avec un potentiel
        client, de comprendre ce que requiert le client, de développer une
        première vue synthétique de ses besoins, puis d'adapter au mieux un
        produit ou service de l'entreprise pour correspondre aux besoins du
        client, d'expliquer voire directement de lui vendre la solution, et
        enfin souvent de rester en contact pour assurer une transition vers les
        équipes techniques et que ces dernières fournissent bien la solution
        vendue. En informatique, du fait de la variété importante de situations
        et solutions possibles, il est fréquent que la personne soit
        effectivement ingénieur de formation ou qu'elle ait fait partie par le
        passé des équipes technique de la même entreprise.
    },
}


\newacronym[description = {filiale d'Orange en cybersécurité}]
    {ocd}{OCD}{Orange Cyberdéfense}

\newacronym[description = {%
    filiale d'Orange proposant des services de télécommunication et
    informatiques pour les entreprises en France et dans le monde
}]{obs}{OBS}{Orange Business Services}

\newacronym[description = {%
    l'ensemble de précautions, vérifications régulières, redondances,
    améliorations, \ldots permettant de rester à un niveau de risques estimé
    acceptable selon les critères de l'organisation concernée
}]{mcs}{MCS}{Maintient en Conditions de Sécurité}

\newacronym[description = {%
    pôle ou équipe chargée de l'implémentation et l'exécution d'un suivi en
    partie automatisé d'un parc informatique pour être en mesure de détecter et
    de remédier aux menaces le concernant le plus rapidement possible
}]{soc}{SOC}{\textit{Security Operation Center}}

\newacronym[description = {%
    réunion de coordination interne à \gls{ocd} en début de projet pour réaliser
    une transition entre les \gls{presales} et les équipes de \gls{ocd-build}
    afin de transmettre les informations importantes et donc de préparer la
    \gls{rle}.
}]{rli}{RLI}{Réunion de Lancement Interne}

\newacronym[description = {%
    réunion de début de projet pour le \gls{ocd-build} suite à la \gls{rli}
    cette fois-ci avec le client présent pour d'abord faire rencontrer les
    différentes personnes des deux côté, puis commencer à préparer une
    spécification précise de l'implémentation à réaliser et le continuer selon
    la taille du projet dans les jours, semaines ou mois suivants avec une série
    d'atelier techniques
}]{rle}{RLE}{Réunion de Lancement Externe}

\newacronym[description = {}]{hld}{HLD}{\textit{High-Level Design}}

\newacronym[description = {}]{lld}{LLD}{\textit{Low-Level Design}}

\newacronym[description = {}]{dsd}{DSD}{Dossier de Spécifications Détaillées}

%\newacronym[description = {}]{}{}{}
