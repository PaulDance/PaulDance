\newglossaryentry{integ}{%
    name = intégration,
    description = {%
        L'intégration de système est une activité d'ingénierie qui consiste en
        le processus de mise en relation de différents systèmes, souvent
        disparates, dans le but qu'ils deviennent composants d'un nouveau
        système plus global. En informatique, il peut s'agir de connexion
        physiques entre machines ou logiques entre logiciels. En sécurité, cela
        ajoute certaines contraintes, notamment de ne pas s'arrêter uniquement
        au bon fonctionnement du système global mais d'inclure aussi le
        non-fonctionnement souhaité de ce qui n'appartient pas au système%
    },
}

\newglossaryentry{ocd-build}{%
    name = \textit{Build},
    description = {%
        ou <<construction>>, c'est-à-dire <<implémentation>>~: sous-équipe de
        l'\gls{integ} chargée d'une partie de la conception de l'architecture à
        mettre en place pour répondre aux besoins d'un client, puis surtout de
        son implémentation effective qui nécessite souvent une intervention
        matérielle et l'établissement de configurations logicielles cibles, pour
        enfin réaliser une migration vers le nouveau système%
    },
}

\newglossaryentry{ocd-run}{%
    name = \textit{Run},
    description = {%
        ou <<exécution>>, c'est-à-dire <<exploitation>>~: sous-équipe de
        l'\gls{integ} chargée d'assurer le bon fonctionnement du système
        implémenté par le \gls{ocd-build} de sorte qu'il soit opérationnel la
        majeure partie du temps, de surveiller la sécurité du parc informatique
        en question et donc réaliser le rôle de \gls{soc}, mais aussi de faire
        évoluer les configurations logicielles au fur et à mesure des demandes
        du client par voie de tickets%
    },
}

\newglossaryentry{presales}{%
    name = avant-ventes,
    description = {%
        ou dans le cas de l'informatique, en particulier la sécurité, aussi
        appelé <<ingénieur avant-vente>> est une personne qui intervient dans le
        procédé de l'acquisition d'un client jusqu'à la vente en elle-même. La
        personne va avoir pour rôle d'entrer en contact avec un potentiel
        client, de comprendre ce que requiert le client, de développer une
        première vue synthétique de ses besoins, puis d'adapter au mieux un
        produit ou service de l'entreprise pour correspondre aux besoins du
        client, d'expliquer voire directement de lui vendre la solution, et
        enfin souvent de rester en contact pour assurer une transition vers les
        équipes techniques et que ces dernières fournissent bien la solution
        vendue. En informatique, du fait de la variété importante de situations
        et solutions possibles, il est fréquent que la personne soit
        effectivement ingénieur de formation ou qu'elle ait fait partie par le
        passé des équipes technique de la même entreprise%
    },
}

\newglossaryentry{staging}{%
    name = \textit{staging},
    description = {%
        ou <<déploiement intermédiaire>> est une phase de tests après les tests
        locaux et avant la production: on parle aussi souvent de
        <<pré-production>>. Il s'agit d'établir un environnement de test qui
        s'approche le plus possible de l'environnement de production effectif.
        L'intérêt est que cela permet de vérifier certaines parties du système,
        comme par exemple l'interaction avec une base de données externe, qui
        sont dans un environnement de test plutôt en local sur le même hôte,
        mais aussi d'être en mesure d'avoir un contrôle plus important de
        l'environnement global, comme par exemple vider la base de données et y
        charger des données de test qui couvre des cas précis%
    },
}

\newglossaryentry{rev-proxy}{%
    name = \textit{reverse proxy},
    description = {%
        est un type de serveur proxy qui s'occupe de récupérer des ressources au
        nom d'un client faisant une requête et auprès d'un ou plusieurs serveurs
        effectifs. Ces ressources sont ensuite renvoyées au client comme si
        elles provenaient du serveur proxy lui-même. Cette technique est surtout
        utilisée pour mettre en place de la répartition de charge%
    },
}

%\newglossaryentry{}{%
%    name = ,
%    description = {},
%}


\newacronym[description = {filiale d'Orange en cybersécurité}]
    {ocd}{OCD}{Orange Cyberdéfense}

\newacronym[description = {%
    filiale d'Orange proposant des services de télécommunication et
    informatiques pour les entreprises en France et dans le monde%
}]{obs}{OBS}{Orange Business Services}

\newacronym[description = {%
    l'ensemble de précautions, vérifications régulières, redondances,
    améliorations, \ldots{} permettant de rester à un niveau de risques estimé
    acceptable selon les critères de l'organisation concernée%
}]{mcs}{MCS}{Maintient en Conditions de Sécurité}

\newacronym[description = {%
    pôle ou équipe chargée de l'implémentation et l'exécution d'un suivi en
    partie automatisé d'un parc informatique pour être en mesure de détecter et
    de remédier aux menaces le concernant le plus rapidement possible%
}]{soc}{SOC}{\textit{Security Operation Center}}

\newacronym[description = {%
    réunion de coordination interne à \gls{ocd} en début de projet pour réaliser
    une transition entre les \gls{presales} et les équipes de \gls{ocd-build}
    afin de transmettre les informations importantes et donc de préparer la
    \gls{rle}%
}]{rli}{RLI}{Réunion de Lancement Interne}

\newacronym[description = {%
    réunion de début de projet pour le \gls{ocd-build} suite à la \gls{rli}
    cette fois-ci avec le client présent pour d'abord faire rencontrer les
    différentes personnes des deux côtés, puis commencer à préparer une
    spécification précise de l'implémentation à réaliser et le continuer selon
    la taille du projet dans les jours, semaines ou mois suivants lors d'une
    série d'ateliers techniques%
}]{rle}{RLE}{Réunion de Lancement Externe}

\newacronym[description = {%
    ou <<Conception de Haut Niveau>> est un dossier regroupant les
    spécifications de l'architecture à implémenter d'un point de vue assez
    global en partant de l'idée générale de départ proposée par les
    \gls{presales}. Il ne s'agit pas de préciser toutes les valeurs effectives
    des options principales de configurations cibles à mettre en place: c'est
    plus le rôle du \gls{lld} qui en découle%
}]{hld}{HLD}{\textit{High-Level Design}}

\newacronym[description = {%
    ou <<Conception de Bas Niveau>> est un dossier regroupant les spécifications
    les plus détaillées possibles de l'architecture à implémenter telles que
    modèles de machine à déployer, versions de systèmes d'exploitation à
    installer, interfaces, adresses, routes, options, autorisations, règles,
    objets, \ldots{} à configurer. Il est en général une continuation directe du
    \gls{hld} associé%
}]{lld}{LLD}{\textit{Low-Level Design}}

\newacronym[description = {%
    dossier très similaire à un \gls{lld}, mais plutôt employé lorsque la taille
    du projet n'est pas trop conséquente et permet donc de rédiger directement
    les spécifications de l'implémentation en un temps pas trop long puisque la
    plupart des sujets abordés lors d'ateliers techniques avec le client ne
    posent pas de problème particulier et ne sont donc pas bloquants%
}]{dsd}{DSD}{Dossier de Spécifications Détaillées}

\newacronym[description = {%
    dossier en général pour un seul site qui regroupe un ensemble de fiches de
    tests d'acceptation à effectuer avec succès pour considérer le déploiement
    du site et la migration globale comme des succès, et ainsi que le site soit
    considéré comme livré au client%
}]{dtv}{DTV}{Dossier de Tests de Validation}

\newacronym[description = {%
    session organisée pour un site et effectuée par le \gls{ocd-build} avec le
    client lors de laquelle tous les tests d'acceptation du \gls{dtv} associé
    sont exécutés et leur résultats respectifs rapportés dans le document%
}]{vabf}{VABF}{Vérification d'Aptitude au Bon Fonctionnement}

\newacronym[description = {%
    ou en anglais <<\textit{runbook}>>, est une compilation systématique des
    procédures et des opérations que l'administrateur ou l'opérateur du système
    effectue. Un tel dossier leur sert aussi souvent de référence, autant pour
    du partage d'informations que de base pour leur travail quotidien. Il peut
    également contenir des descriptions pour le traitement de demandes spéciales
    et d'éventualités, mais aussi de petits changements à réaliser eux-mêmes%
}]{dex}{DEX}{Dossier d'EXploitation}

\newacronym[description = {%
    période suivant une \gls{vabf} complétée lors de laquelle le client utilise
    l'environnement de production cible déployé et fonctionnel du point de vue
    des tests déjà effectués, mais cette fois-ci pour vérifier que cela reste le
    cas dans la durée, c'est-à-dire qu'il n'y ait pas de comportement erratique
    ou d'instabilité seulement à certains moments de la journée par exemple, ou
    lorsque les systèmes sont soumis à des fortes charges, ou encore de temps en
    temps de manière totalement stochastique%
}]{vsr}{VSR}{Vérification de Service Régulier}

\newacronym[description = {pare-feu Fortinet~\cite{fgt}}]{fgt}{FGT}{FortiGate}

\newacronym[description = {%
    appareil de gestion centralisée d'une majeure partie des fonctionnalités des
    produits Fortinet~\cite{fmg}%
}]{fmg}{FMG}{FortiManager}

\newacronym[description = {%
    appareil de récolte de journaux générés par les divers appareils Fortinet
    lui les communiquant~\cite{faz}%
}]{faz}{FAZ}{FortiAnalyzer}

\newacronym[description = {%
    est l'entreprise ferroviaire publique française%
}]{sncf}{SNCF}{Société Nationale des Chemins de fer Français}

\newacronym[description = {%
    ou <<pare-feu d'application web>> est une forme de pare-feu applicatif
    spécifique au Web qui filtre, surveille et bloque du trafic \gls{http}
    allant vers et venant d'un service Web. En inspectant le contenu du trafic
    \gls{http}, un tel pare-feu est en mesure d'empêcher des attaques exploitant
    des vulnérabilités connues dans une application Web, telles que des
    injections SQL, \gls{xss}, inclusion de fichiers et mauvaises configurations
    de système%
}]{waf}{WAF}{\textit{Web Application Firewall}}

\newacronym[description = {%
    catégorie de vulnérabilités typiquement d'applications Web permettant à un
    attaquant l'injection de scripts destinés au côté client dans des pages Web
    vues par d'autres utilisateurs, pouvant ainsi lui servir à contourner
    certaines mesures de contrôle d'accès%
}]{xss}{XSS}{\textit{Cross-Site Scripting}}

\newacronym[description = {%
    ou <<système d'entrave à l'intrusion>> est un type d'appareil ou logiciel
    qui surveille des réseaux ou systèmes pour des signes d'activités
    malveillantes ou violations de politiques centrales, afin d'en faire le
    rapport à un système de gestion d'évènements, mais aussi de prendre des
    mesures dans le but d'en réduire l'impact%
}]{ips}{IPS}{\textit{Intrusion Prevention System}}

\newacronym[description = {%
    ou <<zone démilitarisée>> est un sous-réseau logique ou physique séparé qui
    contient et expose les services informatiques d'une organisation à un réseau
    externe non fiable tel que l'Internet public%
}]{dmz}{DMZ}{\textit{DeMilitarized Zone}}

\newacronym[description = {%
    protocole à la base du Web et habituellement transporté par \acrfull{tcp}%
}]{http}{HTTP}{\textit{HyperText Transfer Protocol}}

\newacronym[description = {%
    protocole de transport en mode connecté fondamental à l'Internet%
}]{tcp}{TCP}{\textit{Transmission Control Protocol}}

\newacronym[description = {%
    protocole de transport en mode déconnecté fondamental à l'Internet%
}]{udp}{UDP}{\textit{User Datagram Protocol}}

\newacronym[description = {%
    est un protocole cryptographique fonctionnant au niveau session et conçu
    pour fournir confidentialité et intégrité aux communications effectuées au
    travers d'un réseau informatique, ainsi que l'authentification des partis.
    Il est le successeur de \gls{ssl}%
}]{tls}{TLS}{\textit{Transport Layer Security}}

\newacronym[description = {%
    prédécesseur maintenant déprécié de \gls{tls}. Le terme est encore beaucoup
    utilisé dans le nom de certaines fonctionnalités proposées dans des produits
    de nombreux fabriquants, surtout par abus de langage%
}]{ssl}{SSL}{\textit{Secure Sockets Layer}}

\newacronym[description = {%
    entreprise française de production et fourniture d'électricité détenue à
    80\% par l'État français%
}]{edf}{EDF}{Électricité de France}

\newacronym[description = {%
    ou <<haute disponibilité>> est une caractéristique d'un système ayant la
    capacité d'assurer sur le long terme un certain niveau de performance
    opérationnelle, souvent mesurée en temps de disponibilité d'un service%
}]{ha}{HA}{\textit{High Availability}}

\newacronym[description = {%
    ou <<réseau local virtuel>> est un type de réseau informatique où la
    partition ne se fait pas tel qu'habituellement à la couche OSI réseau avec
    le protocole IP, mais directement à la couche liaison de données avec le
    protocole Ethernet, ou plus précisément son extension IEEE 802.1q. Cela
    permet de faire apparaître virtuellement plusieurs hôtes comme faisant
    partie de réseaux différents alors qu'en réalité reliés à un même réseau
    physique, souvent le même commutateur%
}]{vlan}{VLAN}{\textit{Virtual Local Area Network}}

\newacronym[description = {%
    offre d'accès Internet d'\gls{obs}, en général une DSL asymétrique (ADSL)%
}]{fi}{FI}{\textit{Flexible Internet}}

\newacronym[description = {%
    offre d'accès Internet d'\gls{obs}, en général une DSL symétrique (SDSL)%
}]{bi}{BI}{\textit{Business Internet}}

%\newacronym[description = {}]{}{}{}
